\section{Lab-01}
\begin{flushleft}
  \textbf{Terminale:} \ace l'ambiente testuale di interazione con il sistema operativo. \par Tipicamente usato come
  applicazione all'interno dell'ambiente grafico ed \ace possibile avviarne pi\acu istanze, \ace anche disponibile all'avvio. 
\end{flushleft}
\begin{flushleft}
  \textbf{Shell}: All'interno del terminale, l'interazione avviene utilizzando una applicazione specifica in esecuzione 
  al suo interno, comunemente detta Shell. \par 
  Essa propone un prompt per l'immissione diretta dei comandi da tastiera e fornisce un feedback testuale, \ace anche possibile 
  eseguire sequenze di comandi preorganizzate contenute in file testuali, chiamati \textit{script} o \textit{batch}.
  A seconda della modalit\aca alcuni comandi possono avere senso o meno o comportarsi in modo particolare.\par \par
  L'insieme dei comandi e delle regole di composizione costituisce un linguaggio di programmazione orientato allo scripting. 
\end{flushleft}
\begin{flushleft}
  \textbf{Bash}: Esistono molte shell. La shell \textbf{Bash} \ace una di quelle pi\acu utilizzate e molte sono comunque sono molto simili 
  tra di loro, ma hanno sempre qualche differenza. \par
  Tipicamente - almeno in sessioni non grafiche - al login un utent ha associata una shell particolare
\end{flushleft}
\begin{flushleft}
  \textbf{POSIX}: \textbf{P}ortable \textbf{O}perating \textbf{S}ystem \textbf{I}nterface for Uni\textbf{x}: \ace una famiglia  di standard IEEE.\\
  Nel caso delle shell in particolare definisce una serie di regole e comportamenti  che possono favorire la probabilit\aca \par 
  La shell bash soddisfa molti requisiti POSIX, ma presenta alcune differenze ed "estensioni" per agevolare la programmazione. 
\end{flushleft}
\begin{flushleft}
  \textbf{Comandi Interattivi}: La shell attende un input dall'utente e al completamento, di solito premendo INVIO, lo elabora. \par 
  Per indicare l’attesa mostra all’utente un PROMPT (può essere modificato). \\
  Fondamentalmente si individuano 3 canali:
  \begin{itemize}
    \item Standard input (tipicamente la tastiera), canale 0, detto \textit{stdin}
    \item Standard output (tipicamente il video), canale 1, detto \textit{stdout}
    \item Standard error (tipicamente il video), canale 2, detto \textit{stderr}
  \end{itemize}
\end{flushleft}
\begin{flushleft}
  \textbf{Struttura generale comandi}: Solitamente un comando \ace identificato da una parola chiave cui possono seguire uno o 
    più "argomenti" opzionali o obbligatori, accompagnati da un valore di riferimento o 
    meno (in questo caso hanno valore di "flag") e di tipo posizionale o nominale. A volte 
    sono ripetibili. \par
    \begin{center}
      \textit{ls -alh /tmp} 
    \end{center}
    gli argomenti nominali sono indicati con un trattino cui segue una voce (stringa 
    alfanumerica) e talvolta presentano una doppia modalit\aca di riferimento: breve 
    (tipicamente voce di un singolo carattere) e lunga (tipicamente un termine mnemonico)
    \begin{flushleft}
      \textit{ls -h} / \textit{ls -help}
    \end{flushleft}
\end{flushleft}
\begin{flushleft}
  \textbf{Termini}:
  \begin{itemize}
    \item l'esecuzione dei comandi avviene "per riga" (in modo diretto quella che si immette fino a INVIO)
    \item un "termine" (istruzione, argomento, opzione, etc.) è solitamente una stringa alfanumerica senza spazi
    \item spaziature multiple sono spesso valide come singole e non sono significative se non per separare i termini 
    \item \ace possibile solitamente usare gli apici singoli/doppi per forzare una sequenza come singolo termine
    \item gli spazi iniziali e finali di una riga collassano 
    \item le righe vuote sono ignorate
  \end{itemize}
\end{flushleft}
\begin{flushleft}
  \textbf{Commenti}: \ac{E} anche possibile utilizzare dei commenti da passare alla shell. L'unico modo formale 
  \ace l'utilizzo del carattere \# per cui esso e tutto ci\aco che segue fino al termine della riga viene ignorato
  \begin{flushleft}
    \textit{ls -la \# this part is a comment}
  \end{flushleft} 
\end{flushleft}
\begin{flushleft}
  \textbf{Comandi fondamentali}: I comandi possono essere "builtin" (funzionalit\aca intrinseche dell'applicazione 
  shell utilizzata) o "esterni" (applicazioni eseguibili che risiedono su disco).
  \begin{itemize}
    \item clear (\textit{clearing the terminal screen in Linux})
    \item pwd (\textit{prints the current working directory path, starting from the root(/)})
    \item ls  (\textit{list files or directories})
    \item cd (\textit{list the contents of the current directory})
    \item wc (\textit{creates a new file in your current working directory with the name wc with the output of your ls command})
    \item date (\textit{displays and sets the system date and time})
    \item cat (\textit{allows us to create single or multiple files, view content of a file, concatenate files and redirect output in terminal or files})
    \item echo (\textit{displaying lines of text or string which are passed as arguments on the command line})
    \item alias/unalias (\textit{is a way to make a complicated command or set of commands simple/remove the definition for each alias name specified})
    \item test (\textit{used as part of the conditional execution of shell commands})
    \item read (\textit{attempts to read up to count bytes from file descriptor fd into the buffer starting at buf})
    \item file (\textit{tells you the type of a file})
    \item chown (\textit{changes the user and/or group ownership of each given file})
    \item chmod (\textit{used to manage file system access permissions on Unix and Unix-like systems})
    \item cp/mv (\textit{copy a file/directory into another directory/moves a file/directory into another directory})
    \item help (\textit{listing all possible commands that are pre-installed in Ubuntu})
    \item type (\textit{used to describe how its argument would be translated if used as commands})
    \item grep (\textit{allows you to find a string in a file or stream})
    \item function (\textit{a reusable block of code})
  \end{itemize}
\end{flushleft}
\begin{flushleft}
  \textbf{Canali Input/Output} \\
  Ogni comando lavora su un insieme di canali, come standar output o un file descriptor 
\end{flushleft}
\begin{flushleft}
  \textbf{Ridirizionamento di base}\par 
  I canali possono anche essere ridirezionati, ad esempio
  \begin{itemize}
    \item ls $1>$/tmp/out.txt $2>$/tmp/err.txt
    \item ls nonExistingItem $1>/tmp/err.txt$ $2>\&1$
  \end{itemize}
  \begin{itemize}
    \item $<$: $command<file$ invia l'input al comando (file.txt read-only) \\
          mail -s "subject" $rcpt< context.txt$ (anzich\ace iterattivo)
    \item $<>$: come sopra solo che il file.txt \ace aperto in read-write
    \item $source>target: command 1>out.txt 2>err.txt$ \\
          ridireziona source su target (source pu\aco essere sottointeso, target pu\aco essere un canale)
    \item $>|$: si comporta come $>$ ma forza la sovrascrittura anche se bloccata nelle configurazioni
    \item $>>$: si comporta come $>$ ma opra un append se la destinazione esiste
    \item $<<$: here-document, consente all'utente di specificare  un terminatore testuale, dopodiche 
          accetta l'input fino alla ricezione di tale terminatore
    \item $<<<$: here-string, consente di fornire input in maniera non iterattiva
  \end{itemize}
  Esistono molte varianti e possibilit\aca di combinazione dei vari operatori \\
  Un caso molto usato \ace la soppressione dell'output (utile per gestire solo side-effects, come 
  il codice di ritorno), esempio:\\
  $type command 1>dev/null 2>\&1$ (per sapere se command esiste)
\end{flushleft}
\begin{flushleft}
  \textbf{Ambiente e variabili}\par 
  \begin{itemize}
    \item La shell pu\aco utilizzare variabili per memorizzare e recuperare i valori 
    \item I valori sono generalmente trattati come stringhe o interi: sono presenti anche semplici array
    \item Il formato del nome \ace del tipo: $^{\wedge}[\_[:alpha:]\,]\,[\_[:alnum:]\,]\*\$$
    \item Per set (impostare) / get (accedere) al valore di una variabile: 
          \begin{itemize}
            \item Il set si effettua con $[export] variabile=valore$, lo scope \ace generalmente quello 
                  del processo attuale: anteponendo export si rende disponibile anche ai processi figli 
            \item Il get si effettua con $\$variabile$ o $\$\{variabile\}$
          \end{itemize}
    \item La shell opera in un ambiente in cui ci sono alcuni riferimenti impostabili e utilizzabili
          attraverso l'uso delle cosiddette "variabili d'ambiente" con cui si intendono generalmente quelle
          con un significato particolare per la shell stessa, tra le variabili d'ambiente pi\acu comuni vi sono:
          $$SHELL, PATH, TERM, PWD, PS1, HOME$$
          Essendo variabili si impostano e usano come le altre
  \end{itemize}
  \textbf{Variabili di sistema} \par 
  Alcune variabili sono impostate e/o utilizzate direttamente dal sistema in casi particolari. 
  \begin{itemize}
    \item SHELL: contiene il riferimento alla shell corrente (path completo)
    \item PATH: contiente i percorsi in ordine di priorit\aca in cui sono cercati i comandi (separati da :)
    \item TERM: contiene il tipo di terminale corrente
    \item PWD: contiene la cartella corrente
    \item PS1: contiene il prompt e si possono usare marcatori speciali
    \item HOME: contiene la cartella principale all'utente corrente
  \end{itemize}
\end{flushleft}
\begin{flushleft}
  \textbf{Esecuzione comandi e \$PATH}\par
  Quando si immette un comando (o una sequenza di comandi) la shell analizza 
  quanto inserito (parsing) e individua le parti che hanno la posizione di comandi 
  da eseguire: se sono interni ne esegue le funzionalità direttamente altrimenti 
  cerca di individuare un corrispondente file eseguibile: questo è normalmente
  cercato nel file-system solo e soltanto nei percorsi definiti dalla variabile PATH a 
  meno che non sia specificato un percorso (relativo o assoluto) nel qual caso viene 
  utilizzato esso direttamente.
  Dall'ultima osservazione discende che per un'azione abbastanza comune come 
  lanciare un file eseguibile (non “installato”) nella cartella corrente occorre 
  qualcosa come: ./nomefile
\end{flushleft}
\begin{flushleft}
  \textbf{Array}\par 
  \begin{itemize}
    \item Definizione: lista=("a" 1 "b" 2 "c" 3) separati da spazi
    \item Output completo: $\{\$lista[@]\}$
    \item Accesso singolo: $\{\$lista[x]\}$
    \item Lista indici: $\{\$!lista[@]\}$
    \item Dimensione: $\{\#lista[@]\}$
    \item Set elemento: $lista[x] = value$
    \item Append: $lista+=(value)$
    \item Sub-array: $\$lista[@]:s:n$, dall'indice s di lunghezza n
  \end{itemize}
\end{flushleft}
\begin{flushleft}
  \textbf{Variabili \$\$ e \$?}\\
  Le variabili \$\$ e \$? non possono essere impostate manualmente.
  \begin{itemize}
    \item \$\$: contiente il PID del processo attuale
    \item \$?: contiente il codice di ritorno dell'ultimo comando eseguito
  \end{itemize}
\end{flushleft}
\begin{flushleft}
  \textbf{Esecuzione comandi e Parsing}\par 
  La riga dei comandi \ace elaborata con una serie di azioni "in sequenza" e poi 
  rielaborata pi\acu volte, tra le azioni ci sono:
  \begin{itemize}
    \item Sostituzioni speciali della shell
    \item Sostituzione variabili
    \item Elaborazione subshell
  \end{itemize}
  sono svolte con un ordine di priorit\aca e poi l'intera riga \ace rielaborata
\end{flushleft}
\begin{flushleft}
  \textbf{Concatenazioni Comandi}\par 
  \ac{E} possibile concatenare pi\acu comandi in un'unica riga in vari modi con effetti differenti 
  \begin{itemize}
    \item comand1 ; comand2: concatenazione semplice, esecuzione in sequenza
    \item comand1 \&\& comand2: concatenazione logica (AND), l'esecuzione procede solo 
          se il comando precedente non fallisce (\textbf{codice ritorno 0})
    \item comand1 || comand2: concatenazione logica (OR), l'esecuzione procede solo 
          se il comando precedente non fallisce (\textbf{codice riotrno NON 0})
    \item comand1 | comand2 : concatenazione con piping
  \end{itemize}
\end{flushleft}
\begin{flushleft}
  \textbf{Operatori di piping (pipe): | e |\&} \par 
  La concatenazione con gli operatori di piping  cattura l'output di un comando e 
  lo passa in input al successivo 
  \begin{itemize}
    \item ls | wc -l: cattura solo stdout
    \item ls |\& wc -l: cattura stdout e stderr
  \end{itemize}
  \textbf{Nota}: il comando ls ha un comportamento atipico: il suo output di base \ace differente 
  a seconda che il comando sia diretto al terminale o a un piping
\end{flushleft}
\begin{flushleft}
  \textbf{Subshell}\par 
  \ac{E} possibile avviare  una subshell, ossia un sotto-ambiente in vari modi, in 
  particolare raggruppando i comandi tra parentesi tonde: ( ... comandi ... )\\
  Spesso si usa il catturare l'output standard della sequenza che viene sostituito letteralmente 
  e rielaborato, ci sono due modi per farlo: \$ (... comandi ... ) oppure $\grave{} \, comandi \,\, \grave{}$
\end{flushleft}
\begin{flushleft}
  \textbf{Esempio}\par 
  $$\texttt{echo "/tmp" > /tmp/tmp.txt ; ls \$(cat /tmp/tmp.txt)}$$
  i comandi sono eseguiti rispettando la sequenza:
  \begin{enumerate}
    \item \texttt{echo "/tmp" > /tmp/temp.txt}: crea un file temporaneo con /tmp
    \item \texttt{ls \$(cat /tmp/tmp.txt)}: \ace prima eseguita la subshell 
          \begin{itemize}
            \item \texttt{cat /tmp/tmp.txt} che genera in stdout \texttt{"/tmp"} e poi con sostituzione:
            \item \texttt{ls /tmp} mostra il contenuto della cartella /tmp
          \end{itemize}
  \end{enumerate}
\end{flushleft}
\begin{flushleft}
  \textbf{Espansione Aritmetica}\par 
  La sintassi base per una subshell \ace da non confondere con l'espansione aritmetica 
  che utilizza le doppie parentesi rotonde, all'interno delle doppie parentesi rotonde
  si possono rappresentare varie espressioni matematiche inclusi assegnamenti e confronti 
  $$\texttt{((a=7))((a++))((a<10))((a = 3>10?1:0))}$$
  $$\texttt{b = \$((c+a))}$$
\end{flushleft}
\begin{flushleft}
  \textbf{Confronti Logici}\par 
  I costrutti fondamentali per i confronti logici sono il comando \texttt{test ...} e i 
  ragruppamenti tra parentesi quadre singole e doppie: \texttt{[ ... ]}, \texttt{[[ ... ]]}\\
  Dove \texttt{test ... } e \texttt{[ ... ]} sono builtin equivalenti mentre 
  \texttt{[[ ... ]]} \ace una coppia di \textbf{shell-keywords} \\
  In tutti i casi il blocco di confronto  genera il codice in uscita 0 in caso di successo, un 
  valore differente altrimenti\par 
  \begin{itemize}
    \item Built-in: sono sostanzialmente dei comandi il cui corpo dell'esecuzione  \ace incluso 
          nell'applicazione shell direttamente (non eseguibili esterni) e quindi seguono sostanzialmente 
          le regole generali dei comandi 
    \item Shell-keywords: sono gestite come marcatori speciali cos\aci che possono 
          "attivare" regole particolari di parsing.
  \end{itemize}
  Un esempio di questo sono gli operatori \texttt{< e >} che normalmente valgono 
  come ridirezionamento, ma all'interno di \texttt{[[ ... ]]} valgono come operatori relazionali\\
  \textbf{Tipologia operatori}\par 
  Le parentesi quadre singole sono POSIX-compilant, mentre le doppie sono un'estensione bash. 
  Nel primo caso gli operatori relazionali tradizionali ($>,<,...$) non possono essere 
  usati perch\ace hanno un altro significato, salvo eventualmente l'utilizzo nelle doppie parentesi quadre, 
  e quindi se ne usano di specifici che hanno per\aco un equivalente pi\acu tradizionale nel secondo caso\\
  Gli operatori e le sintassi variano a seconda del tipo di informazioni utilizzate: c'\ace una 
  distinzione sottile tra confronti per stringhe e per interi.
  \textbf{Confronti tra Interi e Stringhe}
  $$\begin{array}{l|c|c}
    Interi & [...] & [[...]] \\
    \hline & \\
    uguale-a & -eq & == \\
    diverso-da & -ne & != \\
    minore/minore-uguale & -lt / -le & < / <= \\
    maggiore/maggiore-uguale & -gt / -ge & > / >= \\
  \end{array}$$
  $$\begin{array}{l|cc}
    Stringhe & [...] & [[...]] \\
    \hline & \\
    uguale-a & = o == \\
    diverso-da &   != \\
    minore/minore-uguale & /\ < & <  \\
    maggiore/maggiore-uguale & /\ > & >  \\
  \end{array}$$
  Per le strighe nell'uguaglianza bisogna lasciare uno spazio prima e dopo
  \textbf{Confronti tra Operatori Unari}
  Esistono alcuni operatori unari ad esempio per verificare se una stringa \ace 
  vouta, oppure per controllare l'esistenza di file in una cartella
  \begin{itemize}
    \item $[[\texttt{ -f /tmp/prova }]]$: \ace un file?
    \item $[[\texttt{ -e /tmp/prova }]]$: file esiste?
    \item $[[\texttt{ -d /tmp/prova }]]$: \ace una cartella?
  \end{itemize}
  \textbf{N.B.}: \ace possibile utilizzare sia $\left[\right.$ e sia $\left[\left[\right.\right.$\par
  \textbf{Negazione}
  Il carattere ! viene usato per negare i confronti
\end{flushleft}
\begin{flushleft}
  \textbf{Script/Bash} \par 
  \ac{E} possibile raccogliere sequenze di comandi in un file di testo che pu\aco poi 
  essere eseguito:
  \begin{itemize}
    \item Richiamando il tool "bash" e passando il file come argomento
          \begin{center}
            \texttt{bash ../file.sh}
          \end{center}
    \item Impostando il bit "x" e specificando il percorso completo, o solo il 
          nome se la cartella \ace in \$PATH
          \begin{center}
            \texttt{chmod +x ../file.sh \&\& ../file.sh}
          \end{center}
  \end{itemize}
  \textbf{Esempio}: 
  \begin{itemize}
    \item \textbf{Script}: "bashpid.sh"
          \begin{center}
            \texttt{echo \$BASHPID \\ echo \$( \$BASHPID)}
          \end{center}
    \item \textbf{CLI}: \texttt{chmod +x ./bashpid.sh; echo \$BASHPID; ./bashpid.sh}
  \end{itemize}
  \textbf{Elementi particolari}\par 
  Le righe vuote e i commenti (\#) sono ignorati\\
  La prima riga pu\aco essere un \textbf{metacommento}, detto hash-bang, she-bang, che identifica 
  un'applicazione a cui passare il file stesso come argomento (tipicamete usato per identificare l'interprete utilizzato) \\
  Sono disponibili anche variabili speciali:
  \begin{itemize}
    \item \$\@: lista completa degli argomenti passati allo Script
    \item \$\#: numero di argomenti passati allo script 
    \item \$i: i-esimo argomento con i da 0 a n
  \end{itemize}
\end{flushleft}
\begin{flushleft}
  \textbf{Costrutti e Funzioni} \\
  Esistono costrutti come for-loop, if-else, while-loop che possono anche essere scritti 
  su singola riga. \\ 
  \textbf{If-Else}:
  \begin{flushleft}
    \texttt{if [ \$1 -lt 10 ]; then \\
            \tab echo less than 10 \\ 
            elif [ \$1 -ge 20 ]; then \\
            \tab echo greater than 20 \\
            else \\ 
            \tab echo between 10 and 20 \\
            fi}
  \end{flushleft}
  \textbf{For-loop}:
  \begin{flushleft}
    \texttt{for i in \$\{!lista[@]\}; do \\ 
            \tab echo \$\{lista[\$i]\} \\
            done}
  \end{flushleft}
  \textbf{While-loop}: 
  \begin{flushleft}
    \texttt{while [[ \$i < 10 ]]; do \\ 
          \tab echo \$i; (( i++ )) \\
          done}   
  \end{flushleft}
  \textbf{Funzioni} \par 
  \begin{itemize}
    \item \texttt{function\_name () \{ \\
              \tab \underline{local} var1 = 'C' \#variabile con scope locale \\ 
              \tab echo \$1 \$2 \#argomenti normali della funzione \\
              \tab return 44 \#codice di ritorno \\
              \}
          }\\
    \item \texttt{function function\_name () \{commands;\}} \\
    \item \texttt{function\_name arg1 arg2  \#chiamata della funzione con due argomenti }
  \end{itemize}
\end{flushleft}
\begin{flushleft}
  \textbf{Esercizi}
  \begin{enumerate}
    \item Stampare T o F se il valore di input rappresenta un file o una cartella esistente. 
          \begin{center}
            \texttt{[-e \$DATA] \&\& echo "T" || echo "F"}
          \end{center}
    \item Stampare "file", "cartella" o "?" a seconda di cosa l'input rappresenta nel fileStystem.
          \begin{center}
            \texttt{[-f \$DATA] \&\& echo "file" || ([-d \$DATA] \&\& echo "cartella" || echo "?")}
          \end{center}
          \textit{Concatenazione di if uno dentro l'altro}
    \item Stampare il risultato di una semplice operazione aritmetica contenuta nel file 
          rappresentato dal valore di input, oppure "?" se non esiste
          \begin{center}
            \texttt{[-f \$DATA] \&\& echo \$(( \$(cat \$DATA) )) || echo "?"}
          \end{center}
          \textit{Controllo se input \ace un file e computo l'operazione all'interno di esso, tramite (( )), altrimenti 
          ritorna il ?}
    \item Scrivere uno script che dato un qualunque numero di argomenti li restituisca in ordine inverso.
    \item Scrivere uno script che mostri il contenuto della cartella corrente in ordine 
          inverso rispetto all'output generato da "ls", per semplicit\aca assumere l'assenza di spazi 
  \end{enumerate}
  \textbf{N.B.}: Se eseguiti all'interno di uno script, la variabile DATA deve essere esportata con:
  \begin{center}
    \texttt{export DATA = valore}
  \end{center}  
\end{flushleft}
L'utilizzo di BASH, tramite CLI o Script, \ace basilare per poter interagire 
attraverso comandi con il file-system, con le risorse di sistema e per poter invocare 
tools e applicazioni.
