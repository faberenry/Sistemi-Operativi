\section{Lab-02: Docker}
\begin{flushleft}
  \textbf{Docker}: Tecnologia di virtualizzazione a livello del sistema Operativo che 
  consente la creazione, gestione e esecuzione di applicazioni attraverso containers \\
  \textbf{Containers}: sono ambieni leggeri, dinamici ed isolati che vengono eseguiti sopra 
  il kernel Linux\par 
  $$\begin{array}{l|l}
    \textbf{Docker Container} & \textbf{Virtual Machine} \\
    \circ \textit{ Virtualizzazione a livello OS} & \bullet \textit{ Virtualizzazione a livello HW}\\ 
    \circ \textit{ Containers condividono il kernel} & \bullet \textit{ Ogni VM ha il suo OS}\\
    \circ \textit{ Avvio e creazione in secondi } & \bullet \textit{ Avvio e creazione in minuti}\\
    \circ \textit{Leggere (KB/MB)} & \bullet \textit{ Pesanti (GB)}\\
    \circ \textit{Utilizzo leggero di risorse} & \bullet \textit{ Utilizzo intenso di risorse}\\
    \circ \textit{Si distruggono e si rieseguono} & \bullet \textit{ Si trasferiscono}\\ 
    \circ \textit{Minore sicurezza} & \bullet \textit{ Maggiore sicurezza}\\
    & \bullet \textit{ Maggiore controllo} \\
    \circ \textit{Basati su immagini (gi\aca pronte)} & \\
  \end{array}$$
  \textbf{Compatibilit\aca}
  \begin{itemize}
    \item \textbf{Linux}: docker gestisce i containers usando il kernel linux nativo
    \item \textbf{Windows}: docker gestisce i containers usando il kernel linux tramite WSL2
            (originariamente \textbf{virtualizzato} tramite Hyper-V), gestito da un applicazione
    \item \textbf{Mac}: docker gestisce i container usando il kernel linux virtualizzato tramite xhyve hypervysor, 
          gestito da una applicazione
  \end{itemize}
  \textbf{Immagine}: Un'immagine docker \ace un insieme di istruzioni per la creazione di un 
  container. Essa consente di ragruppare varie applicazioni ed eseguirle, con una certa configurazione, 
  in maniera pi\acu rapida attraverso un container \\ 
  \textbf{Container}: I containers sono invece gli ambienti virtualizzati gestiti da docker, che possono essere creati, 
  avviati, fermati ed eliminati. I container sono \textbf{basati} su una immagine
\end{flushleft}
\begin{flushleft}
  \textbf{Gestione dei containers} \par 
  \begin{itemize}
    \item \texttt{docker run [options] <image>}: crea un nuovo container da un'immagine 
    \item \texttt{docker container ls [options]}: mostra i containers attivi (\texttt{[-a]} li mostra tutti)
    \item \texttt{docker exec [options] <container> <command>}: esegue il comando al'interno del container
    \item \texttt{docker stats}: mostra le statistiche di utilizzo dei containers
    \item \texttt{docker <command> --help}: aiuto su specifico comando
  \end{itemize}
  \textbf{Parametri opzionali}
  \begin{itemize}
    \item \texttt{--name <nome>}: assegna un nome specifico al container
    \item \texttt{-d}: detach mode, cio\ace scollega il container
    \item \texttt{-ti}: esegue il container in modalit\aca interattiva, per collegarsi \texttt{docker attach <container>}, scollegarsi \texttt{Crtl-P,Crtl-Q}
    \item \texttt{--rm}: elimina container all'uscita
    \item \texttt{--hostname <nome>}: imposta l'hostname nel container
    \item \texttt{--workdir <path>}: imposta la cartella di lavoro nel container
    \item \texttt{--network host}: collega il container alla rete locale, la modalit\aca host non funziona a causa della VM sottostante
    \item \texttt{--privileged}: esegue il container con i privilegi dell'host
  \end{itemize}
\end{flushleft}
\begin{flushleft}
  \textbf{Gestione delle immagini}\par 
  \ac{E} possibile creare nuove immagini delle docker
  \begin{itemize}
    \item \texttt{docker images}: mostra le immagini salvate
    \item \texttt{docker rmi <imageID>}: elimina immagine se non in uso
    \item \texttt{docker search <keyword>}: cerca un'mmagine nella repository di docker
    \item \texttt{docker commit <container> <repository/imageName>}: crea una nuova immagine dai 
          cambiamenti nel container
  \end{itemize}
\end{flushleft}
\begin{flushleft}
  \textbf{Dockerfile} \par 
  I dockerfile sono dei documenti testuali che raccolgono una serie di comandi 
  necessari alla creazione di una nuova immagine. Ogni nuova immagine sar\aca generata 
  a partire da un'immagine di base, come Ubuntu o l'immagine minimale 'scratch'. La 
  creazione a partire da un docker file viene gestita attraverso del caching che ne 
  permette la ricompilazione rapida in caso di piccoli cambiamenti.\\
  \begin{flushleft}
    \textbf{Esempio}:\\
    \texttt{FROM ubuntu:22.04 \\ 
      RUN apt-get update \&\& apt-get install build-essential nano -y\\
      RUN mkdir /home/labOS \\
      CMD cd /home/labOS \&\& bash
    }\\
    \texttt{docker build -t /labos/ubuntu - < dockerfile}
  \end{flushleft}
\end{flushleft}
\begin{flushleft}
  \textbf{Gestione dei volumi}\\
  Docker salva i file persistenti su bind mount o su dei volumi. Sebbene i bind 
  mount siano strettamente collegati con il filesystem dell'host OS, consentendo 
  dunque una facile comunicazione con il containers, i volumi sono ormai lo standard 
  in quanto indipendenti, facili da gestire e più in linea con la filosofia di docker.
  \begin{flushleft}
    \textbf{Sintassi comandi}
    \begin{itemize}
      \item \texttt{docker volume create <volumeName>}: crea un nuovo volume
      \item \texttt{docker volume ls}: mostra i volumi esistenti
      \item \texttt{docker volume inspect <volumeName>}: esamina volume
      \item \texttt{docker volume rm <volumeName>}: rimuove il volume
      \item \texttt{docker run -v <volume>:<path/in/container> <image>}: crea un 
            nuovo container con il volume specificato montato nel percorso specificato
      \item \texttt{docker run -v <pathHost>:<path/in/container> <image>}: crea un nuovo container
            con un bind mount specificato mostrato nel percorso specificato
    \end{itemize}
  \end{flushleft}
\end{flushleft}
\subsection{GCC}
\begin{flushleft}
  \textbf{GCC} \par 
  GCC = Gnu Compiler Collection, insiemi di strumenti open-source che costituisce 
  lo standard per la creazione di eseguibili Linux. Supporta diversi linguaggi e consente
  la modifica dei vari passaggi intermedi per una completa personalizzazione dell'eseguibile\\
  \textbf{Compilazione} \\
  Gli strumenti gcc possono essere chiamati singolarmente:
  \begin{itemize}
    \item \texttt{gcc -E <sorgente.c> -o <preProcessed.ii|i>}
    \item \texttt{gcc -S <preProcessed.ii|i> -o <assembly.asm|.s>}
    \item \texttt{gcc -c <assembly.asm|.s> -o <objectFile.obj|.o>}
    \item \texttt{gcc <objectFile.obj|.o> -o <executable.out>}
  \end{itemize}
  l'input di ogni comando pu\aco essere il file sorgente, e l'ultimo comando \ace in grado di creare
  direttamente l'eseguibile, inoltre l'assembly e il codice macchina generato dipendono 
  dall'architettura di destinazione
\end{flushleft}
\subsection{Make/Makefile}
\begin{flushleft}
  \textbf{Make}\par 
  \textbf{Make tool}: \ace uno strumento della collezione GNU che pu\aco essere usato 
  per gestire la compilazione automatica e selettiva di grandi e piccoli progetti. \\
  Consente di specificare delle dipendenze tra i vari file, per esempio consentendo solo 
  la compilazione delle librerie in cui i sorgenti sono stati modificati\\ 
  Pu\aco anche essere usato per gestire un deployment di una applicazione, assumendo 
  alcune delle capacit\aca  di uno scipt bash \par 
  \textbf{Makefile}: Make pu\aco eseguire dei makefiles i quali contengono tutte le direttive 
  utili alla compilazione di un'applicazione 
  \begin{center}
    \texttt{make -f makefile}
  \end{center}
  in alternativa, il comando make senza argomenti processer\aca il 'makefile' presente 
  nella cartella di lavoro. \\ 
  Makefiles secondari possono essere inclusi nel makefile principale con delle 
  direttive specifiche, consentendo una gestione pi\acu articolata di grandi progetti
  \begin{flushleft}
    \textbf{Target, prerequisite and recipes}
    \begin{itemize}
      \item Una \textbf{ricetta} o \textbf{regola} è una lista di comandi bash che 
            vengono eseguiti indipendentemente dal resto del makefile.
      \item I \textbf{target} sono generalmente dei files generati da uno specifico insieme di regole. 
            Ogni target pu\aco specificare dei prerequisiti.
      \item \textbf{Prerequisiti}: ovvero degli altri file che devono esistere affinch\ace le regole di un 
            target vengano eseguite. Un prerequisito può essere esso stesso un target! 
    \end{itemize}
    L'esecuzione di un file make inizia specificando uno o più target 
    \texttt{make -f makefile target1 ...} e prosegue a seconda dei vari prerequisiti.\\
    \texttt{target: prerequisite \\
            \tab recipe/rule \\
            \tab recipe/rule \\
            \tab ...\\}
    \textbf{Esempio}:\\
    \texttt{target1: target2 target3 \\
            \tab rule(3)\\
            \tab rule(4) \\
            \tab ...\\
            \tab \\
            target2: target3 \\
            \tab rule(1) \\
            \tab \\ 
            target3:\\
            \tab rule(2)}
    \begin{flushleft}
      \textbf{Sintassi}\par
      Un makefile \ace un file di testo plain in cui righe vuote e parti di testo dal 
      il carattere \# fino alla fine della riga non  in una ricetta (considerato un commento)\\
      Le ricette/rule devono iniziare con il carattere TAB \\
      Una ricetta che inizia con TAB@ non viene visualizzata in output, altrimenti i 
      comandi sono visualizzati e poi eseguiti, una riga con un singolo TAB \ace una ricetta vuota.\\
      Esistono costrutti pi\acu complessi  per necessit\aca particolari. 
    \end{flushleft}
    \begin{flushleft}
      \textbf{Target Speciali}\par 
      Se non viene passato alcun target viene eseguito quello di default: il primo disponibile. 
      Esistono poi dei target con un significato e comportamento speciale 
      \begin{itemize}
        \item \texttt{.INTERMEDIATE e .SECONDARY}: hanno come prerequisiti i target intermedi, nel primo 
              caso sono poi rimossi, nel secondo sono mantenuti a fine esecuzione
        \item \texttt{.PHONY}: ha come prerequisiti i target che non corrispondono a dei files, o 
              comunque da eseguire "sempre" senza verificare l'eventuale file omonimo
      \end{itemize}
      In un target, il simbolo \% sostituisce una qualunque stringa, in un prerequisito
      corrisponde alla stringa sostituita nel target\\
      \textbf{Esempio}:
      \begin{flushleft}
        \texttt{all: ... \\ 
                \tab rule \\ 
                \tab \\ 
                .SECONDARY: target1 .. \\
                \tab \\
                .PHONY: target2 ..\\
                \tab \\ 
                \%.s : \%.c \\ 
                \tab \#prova.s: prova.c \\
                \tab \#src/h.s: src/h.c} 
      \end{flushleft}
    \end{flushleft}
    \begin{flushleft}
      \textbf{Variabile utenti e automaitche}\par 
      \begin{itemize}
        \item \textbf{Utente}: Le variabili utente si definiscono con la sintassi 
              \texttt{nome:=valore} oppure \texttt{nome=valore} e vengono usate 
              con \texttt{\$(nome)}, inoltre possono essere sovrascritte da riga di 
              comando cone \texttt{make nome=value}
        \item \textbf{Automatiche}: Le variabili automatiche possono essere usate all'interno 
              delle regole per riferirsi ad elementi specifici relativi al target corrente. \\
              La variabile SHELL pu\aco essere usata per specificare la shell di riferimento 
              per l'esecuzione di makefile, esempio SHELL=/bin/bash abilita bash anzich\ace SH
      \end{itemize}
    \end{flushleft}
    \begin{flushleft}
      \textbf{Funzioni speciali}\\
      \begin{itemize}
        \item \texttt{\$(eval ...)}:consente di creare nuove regole make dinamiche
        \item \texttt{\$(shell ...)}: cattura l'output di un comando shell
        \item \texttt{\$(wildcard *)}: restituisce un elenco di file che corrispondono alla stringa specificata
      \end{itemize}
      \textbf{Esempio}: \\
      \texttt{LATER=hello\\
        PWD=\$(shell pwd)\\
        OBJ\_FILES:=\$(wildcard *.o)\\
        \tab \\
        target1:\\
        \tab echo \$(LATER) \#hello\\
        \tab \$(eval LATER+= world)\\
        \tab echo \$(LATER) \#hello world
      }
    \end{flushleft}
  \end{flushleft}
  \begin{flushleft}
    \textbf{Esempio}: \\
    \texttt{all: main.out\\
              \tab @echo "Application compiled"\\
            \tab \\
            \%.s: \%.c\\
              \tab gcc -S \$< -o \$@\\
            \tab \\
            \%.out: \%.s\\
              \tab mkdir -p build \\
              \tab gcc \$< -o build/\$@\\
            \tab \\
            clean: \\
              \tab rm -rf build *.out *.s \\
            \tab \\
            .PHONY: clean\\
            \tab \\
            .SECONDARY: make.s\\
    }
  \end{flushleft}
  Docker, GCC e make possono essere utilizzati per la gestione delle varie applicazioni 
  in C. Ognuno di questi strumenti non \ace indispensabile ma permette di creare un 
  flusso di lavoro coerente e strutturato
\end{flushleft}