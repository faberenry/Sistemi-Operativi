\section{Lab-09: Message Queues}
\begin{flushleft}
  \textbf{Message Queues}\par 
  Una coda di messaggi, message queue, \ace una lista concatenata memorizzata 
  all'interno del kernel ed identificata con un ID (un intero positivo univoco), chiamato 
  \textbf{queue identifier}.\\ 
  Questo ID viene condiviso tra i processi interessati, e viene generato attraverso una 
  chiave univoca.\\
  Una coda deve essere innanzitutto generata in maniera analoga ad una FIFO, 
  impostando dei permessi. Ad una coda esistente si possono aggiungere o recuperare 
  messaggi tipicamente in modalit\aca "autosincrona": la lettura attende la presenza di un 
  messaggio, la scrittura attende che via sia spazio disponibile. Questi comportamenti 
  possono per\aco essere configurati.
  \begin{flushleft}
  Quanto trattiamo di message queue, abbiamo due identificativi:
  \begin{itemize}
    \item \textbf{Key}: intero che identifica un insieme di risorse condivisibili nel kernel, come 
          semafori, memoria condivisa e code. Questa chiave univoca deve essere nota a più 
          processi, e viene usata per ottenere il queue identifier.
    \item \textbf{Queue identifier}: id univoco della coda, generato dal kernel ed associato ad una 
          specifica key. Questo ID viene usato per interagire con la coda.
  \end{itemize}
  \end{flushleft}
  \subsection{Creazione}
  \begin{flushleft}
    \textbf{Creazione}\par 
    \begin{flushleft}
      \texttt{int msgget(key\_t key, int msgflg)}
    \end{flushleft}
    Restituisce l'\textbf{identificativo} di una coda basandosi sulla chiave \textbf{key} e sui flags:
    \begin{itemize}
      \item \texttt{IPC\_CREAT}: cra una cosa se non esiste gi\aca, altrimenti restituisce l'identificativo di quella presente
      \item \texttt{IPC\_EXCL}: usato in coppia con CREAT, fallisce se la cosa esiste gi\aca
      \item \texttt{0xxx}: permessi pr accedere alla coda, analogo a quella del file system. utilizzabili anche "S\_IRUSR"
    \end{itemize}
  \end{flushleft}
  \begin{flushleft}
    \textbf{Ottenere una chiave univoca}\par 
    \begin{flushleft}
      \texttt{key\_t ftok(const char *path, int id)}
    \end{flushleft}
    Restituisce una chiave basandosi sul path (una cartella o un file), esistente ed 
    accessibile nel file-system, e sull'id numerico. La chiave dovrebbe essere univoca e 
    sempre la stessa per ogni coppia \textbf{$\tuple{path,id}$} in ogni istante sullo stesso sistema.\\
    Un metodo d'uso, per evitare possibili conflitti, potrebbe essere generare un path (es. 
    un file) temporaneo univoco, usarlo, eventualmente rimuoverlo, ed usare l'id per per 
    rappresentare diverse “categorie” di code, a mo' di indice
    \begin{flushleft}
      \texttt{\#include <sys/ipc.h> \\
              key\_t queue1Key = ftok("/tmp/unique", 1);\\
              key\_t queue2Key = ftok("/tmp/unique", 2);}
    \end{flushleft}
  \end{flushleft}
  \begin{flushleft}
    Esempio creazione:\par
    \texttt{<sys/types.h><sys/ipc.h> <sys/msg.h><stdio.h><fcntl.h>\\
    void main()\{\\
    \halftab remove("/tmp/unique"); //Remove file\\
    \halftab key\_t queue1Key = ftok("/tmp/unique",1); //Get unique key, fail\\
    \halftab creat("/tmp/unique", 0777); //Create file\\
    \halftab queue1Key = ftok("/tmp/unique", 1); //Get unique key $\rightarrow$ ok\\
    \halftab int queueId = msgget(queue1Key, 0777 | IPC\_CREAT); //crt q, ok\\
    \halftab queueId = msgget(queue1Key, 0777); //Get q $\rightarrow$ ok\\
    \halftab msgctl(queue1Key,IPC\_RMID,NULL); //Remove nonexist q $\rightarrow$ fail\\
    \halftab msgctl(queueId,IPC\_RMID,NULL); //Remove q $\rightarrow$ ok\\
    \halftab queueId = msgget(queue1Key, 0777); //Get nonexist q $\rightarrow$ fail\\
    \halftab queueId = msgget(queue1Key, 0777 | IPC\_CREAT); //Create q $\rightarrow$ ok\\
    \halftab queueId = msgget(queue1Key, 0777 | IPC\_CREAT); //Get q $\rightarrow$ ok\\
    \halftab /* Create already existing queue -> fail */\\
    \halftab queueId = msgget(queue1Key , 0777 | IPC\_CREAT | IPC\_EXCL); \\
    \}
    }
  \end{flushleft}
  \textbf{NB: Le queue sono persistenti}
  \subsection{Comunicazione}
  \begin{flushleft}
    \textbf{Comunicazione}\par 
    \begin{flushleft}
      \texttt{struct msg\_buffer\{\\
        \halftab long mtype;\\
        \halftab char mtext[100];\\
        \} message;}
    \end{flushleft}
    Ogni messaggio inserito nella coda ha:
    \begin{itemize}
      \item Un tipo, categoria, rappresentato da un intero maggiore di 0
      \item Una grandezza non negativa
      \item Un payload, un insieme di dati (bytes) di lunghezza corretta
    \end{itemize}
    Al contrario delle FIFO, i messaggi in una coda possono essere recuperati anche 
    sulla base del tipo e non solo del loro ordine "assoluto" di arrivo. Cos\aci come i files, le 
    code sono delle strutture persistenti che continuano ad esistere, assieme ai messaggi 
    in esse salvati, anche alla terminazione del processo che le ha create. L'\textbf{eliminazione} 
    della coda deve essere \textbf{esplicita}.\par 
    Il payload del messaggio non deve essere necessariamente un campo 
    testuale: pu\aco essere una qualsiasi struttura dati. Infatti, un 
    messaggio pu\aco anche essere senza payload.
  \end{flushleft}
  \tab
  \begin{flushleft}
    \textbf{Invio messaggi}\par 
    \begin{flushleft}
      \texttt{int msgsnd(int msqid, const void *msgp, size\_t msgsz, int msgflg);}
    \end{flushleft}
    Aggiunge una \textbf{copia} del messaggio puntato da \textbf{msgp}, con un payload di dimensione 
    \textbf{msgsz}, alla coda identificata da msquid. Il messaggio viene inserito 
    immediatamente se c'\ace abbastanza spazio disponibile, altrimenti la chiamata si blocca 
    fino a che abbastanza spazio diventa disponibile. Se msgflg è IPC\_NOWAIT allora 
    la chiamata fallisce in assenza di spazio. \\
    \textbf{NB}: msgsz \ace la \textbf{grandezza del payload} del messaggio, non del messaggio intero (che 
    contiene anche il tipo)! Per esempio, sizeof(msgp.mtext)
  \end{flushleft}
  \begin{flushleft}
    \textbf{Ricezione}\par 
    \begin{itemize}
      \item \begin{flushleft}
        \texttt{ssize\_t msgrcv(int msqid,void *msgp,size\_t msgsz,\\ 
                \tab \tab  \tab long msgtyp,int msgflg)}
      \end{flushleft}
      Rimuove un messaggio dalla coda msqid e lo salva nel buffer msgp. \\
      msgsz specifica la lunghezza massima del payload del messaggio (per esempio mtext della 
      struttura msgp).\\
      Se il payload ha una lunghezza maggiore e msgflg è MSG\_NOERROR allora il payload viene troncato
      (viene persa la parte in eccesso), se MSG\_NOERROR non \ace specificato allora il payload non viene
      eliminato e la chiamata fallisce.\par
      Se non sono presenti messaggi, la chiamata si blocca in loro attesa. Il flag 
      IPC\_NOWAIT fa fallire la syscall se non sono presenti messaggi
      \item \texttt{ssize\_t msgrcv(int msqid,void *msgp,size\_t msgsz,\\
                    \tab \tab \tab long msgtyp,int msgflg)}\\
            A seconda di msgtyp viene recuperato il messaggio:
            \begin{itemize}
              \item $msgtyp = 0$: primo messaggio della coda (FIFO)
              \item $msgtyp > 0$: primo messaggio di tipo msgtyp, o primo messaggio di tipo 
                    diverso da msgtyp se MSG\_EXCEPT \ace impostato come flag
              \item $msgtyp < 0$: primo messaggio il cui tipo T \ace $min(T \leq |msgtyp|$)
            \end{itemize}                    
    \end{itemize}   
    \textbf{Esempi Comunicazione}
    \begin{enumerate}
      \item \texttt{\#include <sys/types.h><sys/ipc.h><sys/msg.h><string.h><stdio.h> \\
            struct msg\_buffer\{ \\
            \halftab long mtype; \\
            \halftab char mtext[100]; \\
            \} msgp, msgp2; //Two different message buffers\\
            int main(void)\{\\
            \halftab msgp.mtype = 20;\\
            \halftab strcpy(msgp.mtext,"This is a message");\\
            \halftab key\_t q1key = ftok("/tmp/unique", 1);\\
            \halftab int qId = msgget(q1key , 0777 | IPC\_CREAT | IPC\_EXCL);\\
            \halftab int esito = msgsnd(qId , \&msgp, sizeof(msgp.mtext),0);\\
            \halftab esito = msgrcv(qId , \&msgp2, sizeof(msgp2.mtext),20,0);\\
            \halftab printf("Received \%s\n",msgp2.mtext);\\
            \}}
      \item \texttt{\#include <sys/types.h> <sys/ipc.h> <sys/msg.h> <string.h>\\
            typedef struct book\{\\
            \halftab char title[10]; \\
            \halftab char description[200]; \\
            \halftab short chapters;\\
            \} Book;\\
            struct msg\_buffer\{ \\ 
            \halftab long mtype; \\
            \halftab Book mtext; \\
            \} msgp\_snd, msgp\_rcv; //Two different message buffers \\
            int main(void)\{\\
            \halftab msgp\_snd.mtype = 20;\\
            \halftab strcpy(msgp\_snd.mtext.title,"Title");\\
            \halftab strcpy(msgp\_snd.mtext.description,\\ \tab \tab "This is a description");\\
            \halftab msgp\_snd.mtext.chapters = 17;\\
            \halftab key\_t queue1Key = ftok("/tmp/unique", 1);\\
            \halftab int queueId = msgget(queue1Key , 0777 | IPC\_CREAT);\\
            \halftab int esito = msgsnd(queueId, \&msgp\_snd,\\ \tab \tab\tab sizeof(msgp\_snd.mtext),0);\\
            \halftab esito = msgrcv(queueId, \&msgp\_rcv, \\ \tab \tab \tab sizeof(msgp\_rcv.mtext),20,0);\\
            \halftab printf("Received: \%s \%s \%d\n",msgp\_rcv.mtext.title,\\
                    \tab\tab msgp\_rcv.mtext.description, \\ \tab \tab msgp\_rcv.mtext.chapters);\\
            \} }
      \item \texttt{\#include <sys/types.h> <sys/ipc.h> <sys/msg.h> <string.h> \\
      struct msg\_buffer\{  \\
      \halftab long mtype; \\
      \halftab char mtext[100]; \\
      \} msgp\_snd,msgp\_rcv; //Two different message buffers\\
      int main(int argc, char ** argv)\{\\
      \halftab int to\_fetch = atoi(argv[1]); //Input to decide which msg to get\\
      \halftab key\_t queue1Key = ftok("/tmp/unique", 1);\\
      \halftab int queueId = msgget(queue1Key , 0777 | IPC\_CREAT);\\
      \halftab msgp\_snd.mtype = 20;\\
      \halftab strcpy(msgp\_snd.mtext,"A message of type 20");\\
      \halftab int esito = msgsnd(queueId , \&msgp\_snd, sizeof(msgp.mtext),0);\\
      \halftab msgp\_snd.mtype = 10; //Re-use the same message\\
      \halftab strcpy(msgp\_snd.mtext,"Another message of type 10");\\
      \halftab esito = msgsnd(queueId , \&msgp\_snd, sizeof(msgp.mtext),0);\\
      \halftab esito = msgrcv(queueId , \&msgp\_rcv, sizeof(msgp\_rcv.mtext),to\_fetch,0);\\
      \halftab printf("Received: \%s \%s \%d\n",msgp\_rcv.mtext.title,\\
                \tab \tab msgp\_rcv.mtext.description,\\ \tab \tab msgp\_rcv.mtext.chapters);\\
      \} }
    \end{enumerate} 
  \end{flushleft}
  \subsection{Modifica coda}
  \begin{flushleft}
    \textbf{Modificare la coda}\par 
    \begin{flushleft}
      \texttt{int msgctl(int msqid, int cmd, struct msqid\_ds *buf);}
    \end{flushleft}
    Modifica la coda identificata da msqid secondo i comandi cmd, riempiendo buf con 
    informazioni sulla coda (ad esempio timestamp dell'ultima scrittura, dell'ultima 
    lettura, numero messaggi nella coda, etc…). Valori di cmd possono essere:
    \begin{itemize}
      \item IPC\_STAT: recupera informazioni da kernel 
      \item IPC\_SET: imposta alcuni parametri a seconda di buf
      \item IPC\_RMID: rimuove immediatamente la coda
      \item IPC\_INFO: recupera informazioni generali sui limiti delle code nel sistema
      \item MSG\_INFO: come IPC\_INFO ma con informazioni differenti
      \item MSG\_STAT: come IPC\_STAT ma con informazioni differenti
    \end{itemize}
    \begin{flushleft}
      \textbf{Struttura MSQID\_DS}\par 
      \texttt{struct msqid\_ds \{\\
              \halftab struct ipc\_perm msg\_perm; /* Ownership and permissions */\\
              \halftab time\_t msg\_stime; /* Time of last msgsnd(2) */\\
              \halftab time\_t msg\_rtime; /* Time of last msgrcv(2) */\\
              \halftab time\_t msg\_ctime; //Time of creation or last modifi by msgctl\\
              \halftab unsigned long msg\_cbytes; /* \# of bytes in queue */\\
              \halftab msgqnum\_t msg\_qnum; /* \# of messages in queue */\\
              \halftab msglen\_t msg\_qbytes; /* Maximum \# of bytes in queue */\\
              \halftab pid\_t msg\_lspid; /* PID of last msgsnd(2) */\\
              \halftab pid\_t msg\_lrpid; /* PID of last msgrcv(2) */\\
        \};}
    \end{flushleft}
    \begin{flushleft}
      \textbf{Struttura IPC\_PERM}\par 
      \texttt{struct ipc\_perm \{ \\
              \halftab key\_t \_\_key; /* Key supplied to msgget(2) */\\
              \halftab uid\_t uid; /* Effective UID of owner */\\
              \halftab gid\_t gid; /* Effective GID of owner */\\
              \halftab uid\_t cuid; /* Effective UID of creator */\\
              \halftab gid\_t cgid; /* Effective GID of creator */\\
              \halftab unsigned short mode; /* Permissions */\\
              \halftab unsigned short \_\_seq; /* Sequence number */\\
        \};}
    \end{flushleft}
    \begin{flushleft}
      \textbf{Esempio modifica}\par 
      \texttt{int main(void)\{ \\
        \halftab struct msqid\_ds mod;\\
        \halftab int esito = open("/tmp/unique",O\_CREAT,0777);\\
        \halftab key\_t queue1Key = ftok("/tmp/unique", 1);\\
        \halftab int queueId = msgget(queue1Key , IPC\_CREAT | S\_IRWXU );\\
        \halftab msgctl(queueId,IPC\_RMID,NULL);\\
        \halftab queueId = msgget(queue1Key , IPC\_CREAT | S\_IRWXU ); \\
        \halftab esito = msgctl(queueId,IPC\_STAT,\&mod); //Get info on queue\\
        \halftab printf("Current permission on queue: \%d\n",mod.msg\_perm.mode);\\
        \halftab mod.msg\_perm.mode = 0000;\\
        \halftab esito = msgctl(queueId,IPC\_SET,\&mod); //Modify queue\\
        \halftab printf("Current permission on queue: 
                        \\ \tab \tab \%d\n\n",mod.msg\_perm.mode);\\
        \} }
    \end{flushleft}
  \end{flushleft}
  \textbf{NB: Le code sono un comodo metodo di comunicazione per l'invio e la ricezione di 
  informazioni anche "complesse" tra generici}
\end{flushleft}