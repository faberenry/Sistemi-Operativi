\section{Lab-03}
\begin{flushleft}
  \textbf{C Fundamentals}\par 
  \begin{itemize}
    \item \textbf{Tipi e Casting}\par 
          C \ace un linguaggio debolmente e tipizzato che utilizza 8 tipi 
          fondamentali. \ac{E} possibile fare il casting tra tipi 
          differenti: 
          \begin{flushleft}
            \texttt{float a = 3.5; \\
                    int b = (int)a}
          \end{flushleft}
          La grandezza delle variabili \ace dipendente dall'
          architettura di riferimento i valori massimi per ogni 
          tipo cambiano a seconda se la variabile \ace signed o unsigned.
          $$\begin{array}{l|l}
            void = 0 byte & float = 4 bytes \\
            char = 1 byte & long = 8 bytes\\
            short = 2 bytes & double = 8 bytes \\
            int = 4 bytes & long double = 16 bytes \\
          \end{array}$$
    \item \textbf{sizeof}: Si tratta di un operatore che elabora il tipo passato come argomento (tra parentesi) o 
    quello dell’espressione e restituisce il numero di bytes occupati in memoria
    \item \textbf{Puntatori}\\
          C si evolve attorno all'uso di puntatori, ovvero degli 
          alias per zone di memorie condivise tra diverse 
          variabili/funzioni. L'uso di puntatori \ace abilitato da 
          due operatori: * ed \&.
          \begin{itemize}
            \item \textbf{Variabili}:\\
                  *: ha significati diversi a seconda se usato in una dichiarazione 
                  o in una assegnazione\\
                  \texttt{int *pointer; // crea un puntatore ad intero \\
                          int valore = *(pointer) // ottiene un valore puntato}\\
                  \&: ottiene l'indirizzo di memoria in cui \ace collocata 
                  una certa variabile.\\    
                  \texttt{long whereIsValore = \&valore}              
            \item \textbf{Funzioni}: C consente anche di creare dei puntatori a delle 
                  funzioni: puntatori che possono contenere l'indirizzo di funzioni differenti\\
                  \texttt{float (*punt)(float,float)}\\
                  \texttt{ret\_type (*ptnName)(argType, argType,....)}
          \end{itemize}
    \item \textbf{main.c}\par 
          A parte in casi particolari l'applicazione deve avere una funzione "main" che 
          \ace utilizzata come punto di ingresso, il valore di ritorno \ace un int che rappresenta 
          il codice di uscita dell'applicazione (0 se omesso). \\
          Quando viene invocata riceve in input il numero di argomenti, \texttt{int argc}, in cui \ace incluso 
          il nome dell'eseguibile e la lista degli argomenti, \texttt{char * argv[]}, come vettore 
          di stringhe (vettore di vettori di caratteri)\\
          \textbf{Esempio}: 
          \begin{flushleft}
            \texttt{gcc main.c -o main.o \\
                    ./main.o arg1 arg2}\\
                    In questo esempio il valore di argc = 3, e argv \ace della 
                    forma "0--./main.o; 1-- arg1; 2-- arg2"
          \end{flushleft}
    \item \textbf{printf/fprintf} \\
          \begin{flushleft}
            \texttt{int printf(const char *format, ...)\\
            int fprintf(FILE *stream, const char *format, ...)}
          \end{flushleft}
          Inviano dati sul canale stdout (printf) o su quello specificato (fprintf) secondo il formato 
          indicato.
          Il formato è una stringa contenente contenuti stampabili (testo, a capo, …) ed eventuali 
          segnaposto identificabili dal formato generale:
          \begin{flushleft}
            \texttt{\%[flags][width][.precision][length]specifier}
          \end{flushleft}
          \textbf{Esempio}: \%d (intero con segno), \%c (carattere), \%s (stringa), \dots\\
          Ad ogni segnaposto deve corrispondere un ulteriore argomento del tipo corretto
    \item \textbf{Direttive}\par 
          Il compilatore, nella fase di preprocessing, elabora tutte le direttive presenti nel 
          sorgente. Ogni direttiva viene introdotta con "\#" e può essere di vari tipi:
          $$\begin{array}{l|l}
            \texttt{\#include <lib>} & \textit{copia il contenuto del file lib nel file corrente}\\
            \hline 
            \texttt{\#include "lib"} & \textit{come sopra ma cerca prima nella cartella corrente}\\
            \hline
            \texttt{\#define VAR VAL} & \textit{crea una costante VAR con il contenuto VAL}\\
              &\textit{e sostituisce ogni occorrenza di VAR con VAL}\\
              \hline 
            \texttt{\#define MUL(A,B) A*B} & \textit{dichiara una funzione con parametri A e B,}\\
            &\textit{ queste funzioni hanno una sintassi limitata}\\
            \hline 
            \texttt{\#ifdef, \#ifndef, \#if,}& \textit{rende l'inclusione di parte di codice }\\
            \texttt{\#else, \#endif} & \textit{dipendente da una condizione}\\
          \end{array}$$
          Macro possono essere passate a GCC con -D NAME=VALUE \\
          \textbf{Esempio}: 
          \begin{flushleft}
            \texttt{\#include <stdio.h>\\
                    \#define ITER 5\\
                    \#define POW(A) A*A\\
                    int main(int argc, char **argv) \{ \\
                    \#ifdef DEBUG \\
                    \tab printf("\%d\textbackslash n, argc");\\
                    \tab printf("\%s\textbackslash n", argv[0]);\\
                    \#endif\\
                    \tab int res = 1;\\
                    \tab for (int i = 0; i < ITER; i++)\{ \\
                      \tab \tab res *= POW(argc);\\
                    \tab \} \\
                    \tab return res;\\
                    \} \\
            }
            \texttt{gcc main.c -o main.out -D DEBUG=0\\
            gcc main.c -o main.out -D DEBUG=1\\
            ./main.out 1 2 3 4\\}
            Danno lo stesso risultato 
          \end{flushleft}
          \begin{flushleft}
            \texttt{\#include <stdlib.h>\\
            \#include <stdio.h>\\
            \#define DIVIDENDO 3\\
            int division(int var1, int var2, int * result)\{ \\
            \tab *result = var1/var2;\\
            \tab return 0;\\
            \} \\
            int main(int argc, char * argv[])\{ \\
            \tab float var1 = atof(argv[1]);\\
            \tab float result = 0;\\
            \tab division((int)var1,DIVIDENDO,(int *)\&result);\\
            \tab printf("\%d \textbackslash n",(int)result);\\
            \tab return 0;\\
            \} \\}
          \end{flushleft}
    \item \textbf{Librerie standard}: \par 
          Librerie possono essere usate attraverso la direttiva \#include.
          Tra le più importanti vi sono:
          \begin{itemize}
            \item stdio.h: FILE, EOF, stderr, stdout, stdin, \dots 
            \item stdlib.h: atof(),atoi(),\dots
            \item string.h: memset(), memcpy(), strncat(), \dots 
            \item math.h: sqrt(), sin(), cos(),\dots 
            \item unistd.h: read(), write(), fork(), \dots 
            \item fcntl.h: creat(), open(), \dots 
          \end{itemize}
    \item \textbf{Struct e Unions}: Structs permettono di aggregare diverse variabili, mentre le unions permettono di 
          creare dei tipi generici che possono ospitare uno di vari tipi specificati.\\
          \textbf{Esempio:}\\
            \texttt{struct Books \{ \\
            \tab char author[50]; \\
            \tab char title[50]; \\
            \tab int bookID; \\
            \} book1, book2; } \\
            \texttt{union Result\{ \\
              \tab int intero;\\
              \tab float decimale;\\
              \} result1, result2; } 
    \item \textbf{Typedef}: consente la definizione di nuovi tipi di variabili o funzioni
            \textbf{Esempio}:
            \begin{flushleft}
              \texttt{typedef unsigned int intero; \\
                typedef struct Books\{ \\
                \dots \\
                \} bookType; \\
                intero var = 22; //= unsigned int var = 22; \\
                bookType book1; //= struct Books book1;\\
              } 
            \end{flushleft}
    \item \textbf{Exit}: \texttt{void exit(int status)}\\
          Il processo \ace terminato restituendo il valore status come codice di uscita. Si ottiene 
          lo stesso effetto se all'interno della funzione main si ha return status.
          La funzione non ha un valore di ritorno proprio perch\ace non sono eseguite ulteriori 
          istruzioni dopo di essa.\\
          Il processo chiamante \ace informato della terminazione tramite un segnale apposito. 
    \item \textbf{Vettori e Stringhe} \par 
          \begin{itemize}
            \item \textbf{Vettori}: \par I vettori sono sequenze di elementi omogenei (tipicamente liste di dati dello stesso 
            tipo, ad esempio liste di interi o di caratteri), si realizzano con un puntatore al primo elemento della lista.\\
            \textbf{Esempio}: Con int arr[4] = \{2, 0, 2, 1\} si dichiara un vettore di 4 interi 
            inizializzandolo: sono riservate 4 aree di memoria consecutive di dimensione pari a 
            quella richiesta per ogni singolo intero (tipicamente 2 bytes, quindi 4*2=8 in tutto) \\ 
            \textbf{Esempio}: \texttt{char str[7] = \{'c', 'i', 'a', 'o', 56,57,0\}} : 7*1 = 7 bytes\\
            str \ace dunque un puntatore a char (al primo elemento) e si ha che: \\
            \texttt{str[n] corrisponde a *(str+n)}\\
            e in particolare \\
            \texttt{str[0] corrisponde a *(str+0)=*(str)=*str}\\
            \item \textbf{Stringhe}:\par 
            Le stringhe in C sono vettori di caratteri, ossia puntatori a sequenze di bytes, la cui 
            terminazione \ace definita dal valore convenzionale 0 (zero).\\
            Un carattere tra apici singoli equivale all'intero del codice corrispondente. \\
            In particolare un vettore di stringhe \ace un vettore di vettore di caratteri e dunque:
           \begin{flushleft}
              \texttt{char c; carattere} \\
              \texttt{char * str; vettore di a caratteri o stringa}  \\
              \texttt{char **strarr; vettore di vettore di caratteri o vettore di stringhe} \\
           \end{flushleft}
            Si comprende quindi la segnatura della funzione main con **argv. \\
            C supporta l'uso di stringhe che, tuttavia, corrispondono a degli array di caratteri.\\
            Gli array sono generalmente di dimensione statica e non possono essere ingranditi 
            durante l'esecuzione del programma. Per array dinamici dovranno essere usati 
            costrutti particolari (come malloc). \\
            Le stringhe, quando acquisite in input o dichiarate con la sintassi "stringa", 
            terminano con il carattere '\textbackslash 0' e sono dunque di grandezza \texttt{str\_len+1} \\
            Sebbene ci siano diversi modi per dichiarare ed inizializzare una stringa, questi hanno comportamenti diversi:
            \begin{flushleft}
              \texttt{char string[] = “ciao”; // writable string in the stack \\
              char * string2 = “ciao; // READ-ONLY string \\
              string[2] = ‘a’;\\
              string2[2] = ‘a’; //Segmentation fault! \\}
            \end{flushleft}
            Dato che le stringhe sono riferite con un puntatore al primo carattere non ha senso 
            fare assegnamenti e confronti diretti, ma si devono usare delle funzioni. La libreria 
            standard string.h ne definisce alcune come ad esempio:
            \begin{itemize}
              \item \texttt{char * strcat(char *dest, const char *src)}: aggiunge src in coda a dest
              \item \texttt{char * strchr(const char *str, int c)}: cerca la prima occorrenza di c in str
              \item \texttt{int strcmp(const char *str1, const char *str2)}: confronta str1 con str2
              \item \texttt{size\_t\ strlen(const char *str)}: calcola la lunghezza di str
              \item \texttt{char * strcpy(char *dest, const char *src)}: copia la stringa src in dst
              \item \texttt{char * strncpy(char *dest, const char *src, size\_t n)}: copia n caratteri 
                    dalla stringa src in dst
            \end{itemize}
          \end{itemize}
  \end{itemize}
\end{flushleft}