\section{Lab-06: Segnali}
\begin{flushleft}
  \textbf{Segnali}\par 
  Ci sono vari eventi che possono avvenire in maniera asincrona al normale flusso di un 
  programma, alcuni dei quali in maniera inaspettata e non predicibile. \\ 
  Per esempio, durante l'esecuzione di un programma ci pu\aco essere una richiesta di terminazione o 
  di sospensione da parte di un utente, la terminazione di un processo figlio o un errore 
  generico.\par 
  Unix prevede la gestione di questi eventi attraverso i \textbf{segnali}: quando il sistema 
  operativo si accorge di un certo evento, genera un segnale da mandare al processo 
  interessato il quale potr\aca decidere (nella maggior parte dei casi) come comportarsi.\par
  Il numero dei segnali disponibili cambia a seconda del sistema operativo, con Linux 
  che ne definisce 32. Ad ogni segnale corrisponde sia un valore numerico che 
  un’etichetta mnemonica (definita nella libraria "signal.h") nel formato \texttt{SIGXXX}
  \begin{center}
    $$\begin{array}{l|l}
      SIGALRM (\textit{alarm clock}) & SIGQUIT (\textit{terminal quit})\\
      SIGCHLD (\textit{child terminated}) & SIGSTOP (\textit{stop})\\
      SIGCONT (\textit{continue, if stopped}) & SIGTERM (\textit{termination}) \\
      SIGINT (\textit{terminal interrupt/CTRL + C }) & SIGUSR1 (\textit{user signal}) \\
      SIGKILL (\textit{kill process}) & SIGUSR2 (\textit{user signal}) \\
    \end{array}$$
  \end{center}
  Per ogni processo, all'interno della process table, vengono mantenute due liste:
  \begin{itemize}
    \item \textbf{Pending signals}: segnali emessi dal kernel e che il processo deve ancora gestire.
    \item \textbf{Blocked signals}: segnali che non devono essere comunicati al processo. 
          Chiamata anche con il termine \textbf{signal mask}, maschera dei segnali.
  \end{itemize}
  Ad ogni schedulazione del processo le due liste vengono controllate per consentire al 
  processo di reagire nella maniera pi\acu adeguata\par 
  \begin{flushleft}
    \textbf{Gestione}\\
    I segnali sono anche detti "software interrupts" perch\ace sono, a tutti gli effetti, delle 
    interruzioni del normale flusso del processo generate dal sistema operativo (invece 
    che dall'hardware, come per gli hardware interrupts).\\
    Come per gli interrupts, il programma pu\aco decidere come gestire l'arrivo di un 
    segnale (presente nella lista pending):
    \begin{itemize}
      \item Eseguendo l'azione default
      \item Ignorandolo (non sempre possibile) $\Rightarrow$ programma prosegue normalmente.
      \item Eseguendo un handler personalizzato $\Rightarrow$ programma si interrompe.
    \end{itemize}
    NB: nella pratica, il programma comunica al kernel come vuole che il segnale venga 
    gestito, ed \ace poi il kernel che richiamer\aca la funzione adeguata del programma
  \end{flushleft}
  \begin{flushleft}
    \textbf{Default handler}\\
    Ogni segnale ha un suo handler di default che tipicamente pu\aco:
    \begin{itemize}
      \item Ignorare il segnale
      \item Terminare il processo
      \item Continuare l'esecuzione (se il processo era in stop)
      \item Stoppare il processo
    \end{itemize}
    Ogni processo pu\aco sostituire il gestore di default con una funzione "custom" (a parte 
    per SIGKILL e SIGSTOP) e comportarsi di conseguenza. La sostituzione avviene 
    tramite la system call signal() (definita in "signal.h").
  \end{flushleft}
  \begin{flushleft}
    \textbf{signal()}\par 
    \texttt{sighandler\_t signal(int signum, sighandler\_t handler);}\\
    Imposta un nuovo signal handler handler per il segnale signum. Restituisce il 
    signal handler precedente. Quello nuovo può essere:
    \begin{itemize}
      \item SIG\_DFL: handler di default
      \item SIG\_IGN: ignora il segnale
      \item \texttt{typedef void (*sighandler\_t)(int)}: custom handler
    \end{itemize}
    Esempio: \\
    \texttt{\#include <signal.h> <stdio.h> <stdlib.h>\\
    void main()\{\\
    \halftab signal(SIGINT,SIG\_IGN); //Ignore signal\\
    \halftab signal(SIGCHLD,SIG\_DFL); //Use default handler\\
    \}}
  \end{flushleft}
  \subsection{Handler custom}
  \begin{flushleft}
    \textbf{Custom handler}\par 
    Un custom handler \textbf{deve} essere una funzione di tipo void che accetta come 
    argomento un int rappresentante il segnale catturato. Questo consente allo stesso 
    handlers di gestire segnali diversi.\par 
    \texttt{void myHandler(int par)}\par 
    Esempio: \\
    \texttt{\#include <signal.h> <stdio.h> \\
    void myHandler(int sigNum)\{\\
    \halftab if(sigNum == SIGINT) printf("CTRL+C\n");\\
    \halftab else if(sigNum == SIGTSTP) printf("CTRL+Z\n");\\
    \} \\
    signal(SIGINT,myHandler); \\
    signal(SIGTSTP,myHandler);\\}
  \end{flushleft}
  \begin{flushleft}
    \textbf{signal() return}\par 
    signal() restituisce un riferimento all’handler che era precedentemente assegnato al 
    segnale: 
    \begin{itemize}
      \item NULL: handler precedente era l'handler di default
      \item 1: l'handler precedente era SIG\_IGN
      \item $<address>$: l'handler precedente era *(address)
    \end{itemize}
    Esempio: \\
    \texttt{\#include <signal.h> <stdio.h> \\
    void myHandler(int sigNum)\{\}\\
    int main()\{\\
    \halftab printf("DFL: \%p\n",signal(SIGINT,SIG\_IGN));\\
    \halftab printf("IGN: \%p\n",signal(SIGINT,myHandler));\\
    \halftab printf("Custom:\%p == \%p\n",signal(SIGINT,SIG\_DFL),myHandler);\\
    \}}
  \end{flushleft}
  \begin{flushleft}
    \begin{tabular}{lll}
      SIGXXX & description & default\\
      SIGALRM & (alarm clock) & quit\\
      SIGCHLD & (child terminated) & ignore\\
      SIGCONT & (continue, if stopped) & ignore\\
      SIGINT & (terminal interrupt, CTRL + C)& quit\\
      SIGKILL & (kill process) &quit\\
      SIGSYS & (bad argument to syscall) & quit with dump\\
      SIGTERM & (software termination) & quit\\
      SIGUSR1/2 & (user signal 1/2) & quit\\
      SIGSTOP & (stopped) & quit\\
      SIGTSTP & (terminal stop, CTRL + Z) & quit\\
    \end{tabular}
  \end{flushleft}
  \begin{flushleft}
    \textbf{Esempio}\\
    \texttt{\#include <signal.h> <stdio.h> <unistd.h> <sys/wait.h> \\
    void myHandler(int sigNum)\{\\
    \halftab printf("Child terminated Received \%d\n",sigNum);\\
    \} \\
    int main()\{ \\
    \halftab signal(SIGCHLD,myHandler);\\
    \halftab int child = fork();\\
    \halftab if(!child)\{\\
    \tab return 0; //terminate child\\
    \halftab \} \\
    \halftab while(wait(NULL)>0);\\
    \}}
  \end{flushleft}
  \begin{flushleft}
    \textbf{Inviare i segnali: kill()}\par 
    \texttt{int kill(pid\_t pid, int sig);}\\
    \texttt{\$ kill -$<signo>$ $<pid\_t>$} \\
    Invia un segnale ad uno o più processi a seconda dell'argomento pid:
    \begin{itemize}
      \item pid $>$ 0: segnale al processo con PID=pid
      \item pid = 0: segnale ad ogni processo dello stesso gruppo
      \item pid = -1: segnale ad ogni processo possibile (stesso UID/RUID)
      \item pid $<$ -1: segnale ad ogni processo del gruppo $|$pid$|$
    \end{itemize}
    Restituisce 0 se il segnale viene inviato, -1 in caso di errore.\\
    Ogni tipo di segnale può essere inviato, non deve essere necessariamente un segnale 
    corrispondente ad un evento effettivamente avvenuto \par 
    Esempio \\
    \texttt{\#include <signal.h> <stdio.h> <stdlib.h> <sys/wait.h> <unistd.h> \\
    void myHandler(int sigNum)\{\\
      \halftab printf("[\%d]ALARM!\n",getpid());\\ \}\\
    int main(void)\{ \\
    \halftab signal(SIGALRM,myHandler);\\
    \halftab int child = fork();\\
    \halftab if (!child) while(1); // block the child\\
    \halftab printf("[\%d]sending alarm to \%d in 3 s\n",getpid(),child);\\
    \halftab sleep(3);\\
    \halftab kill(child,SIGALRM); // send ALARM, child's handler reacts\\
    \halftab printf("[\%d]sending SIGTERM to \%d in 3 s\n",getpid(),child);\\
    \halftab sleep(3);\\
    \halftab kill(child,SIGTERM); // send TERM: default is to terminate\\
    \halftab while(wait(NULL)>0);\\
    \}}\par 
    \textbf{kill() da bash}\\
    kill \ace anche un programma in bash che accetta come primo argomento il tipo di 
    segnale (\textbf{kill -l} per la lista) e come secondo argomento il PID del processo\\
    Esempio\\
    \texttt{\#include <signal.h> <stdio.h> <stdlib.h> <unistd.h> \\
    void myHandler(int sigNum)\{ \\
    \halftab printf("[\%d]ALARM!\n",getpid()); \\
    \halftab exit(0);\\
    \} \\
    int main()\{ \\
    \halftab signal(SIGALRM,myHandler);\\
    \halftab printf("I am \%d\n",getpid());\\
    \halftab while(1);\\
    \}}\par 
    \begin{flushleft}
      \texttt{gcc bash.c -o bash.out\\
      \$ ./bash.out\\
      \# On new window/terminal\\
      \$ kill -14 <PID>\\}
    \end{flushleft}
  \end{flushleft}
  \begin{flushleft}
    \textbf{alarm()}\par 
    \texttt{unsigned int alarm(unsigned int seconds);}\\
    Genera un segnale SIGALRM per il processo corrente dopo un lasso di tempo 
    specificato in secondi. Restituisce i secondi rimanenti all'alarm precedente. \\
    Esempio \\
    \texttt{\#include <signal.h> <stdio.h> <stdlib.h> <unistd.h>\\
            short cnt = 0;\\
            void myHandler(int sigNum)\{printf("ALARM!\n"); cnt++;\} \\
            int main()\{ \\
            \halftab signal(SIGALRM,myHandler);\\
            \halftab alarm(0); //Clear any pending alarm\\
            \halftab alarm(5); //Set alarm in 5 seconds\\
            \halftab //Set new alarm (cancelling previous one)\\
            \halftab printf("Seconds remaining to previous alarm \%d\n",alarm(2));\\
            \halftab while(cnt<1);\\
            \}}
  \end{flushleft}
  \begin{flushleft}
    \textbf{pause()}\par
    \texttt{int pause();}\\
    Esempio:\\
    \texttt{\#include <signal.h> <unistd.h> <stdio.h> \\
            void myHandler(int sigNum)\{\\
            \halftab printf("Continue!\n"); \\
            \}\\
            int main()\{\\
            \halftab signal(SIGCONT,myHandler);\\
            \halftab signal(SIGUSR1,myHandler);\\
            \halftab pause();\\
            \}}
  \end{flushleft}
  \begin{flushleft}
    \textbf{Bloccare segnali}\par 
    Oltre alla lista dei "pending signal" esiste la \textbf{lista dei "blocked signals"}, ovvero dei 
    segnali ricevuti dal processo ma volutamente non gestiti. Mentre i segnali ignorati 
    non saranno mai gestiti, i segnali bloccati sono solo \textbf{temporaneamente} non gestiti.\par
    \begin{flushleft}
      Al contrario dei segnali ignorati, un segnale bloccato rimane nello stato pending fino 
      a quando esso non viene gestito oppure il suo handler tramutato in ignore. 
    \end{flushleft}
    Per allertare il kernel dei segnali che devono essere bloccati, un processo pu\aco
    \textbf{modificare la propria signal mask}, ovvero una struttura dati mantenuta nel kernel 
    che può essere alterata con la funzione \texttt{sigprocmask()}
    \begin{flushleft}
      \textbf{Bloccare i segnali: sigset\_t}\par
      La signal mask viene memorizzata come struttura dati ottimizzata che non pu\aco
      essere modificata direttamente. Invece, pu\aco essere \textbf{gestita attraverso un sigset\_t}, 
      cio\ace una struttura dati locale contenente un elenco di segnali.\\
      Questo insieme pu\aco essere modificato con funzioni dedicate e poi 
      pu\aco essere utilizzato per modificare la maschera di segnale stessa.
      \begin{itemize}
        \item \texttt{int sigemptyset(sigset\_t *set); Svuota}
        \item \texttt{int sigfillset(sigset\_t *set); Riempie}
        \item \texttt{int sigaddset(sigset\_t *set, int signo); Aggiunge singolo}
        \item \texttt{int sigdelset(sigset\_t *set, int signo); Rimuove singolo}
        \item \texttt{int sigismember(const sigset\_t *set, int signo); Interpella}
      \end{itemize}
      NB: la modifica di questa struttura non modifica implicitamente la maschera dei 
      segnali! Le modifiche devono essere salvate con \texttt{sigprocmask()}.
    \end{flushleft}
    \begin{flushleft}
      \textbf{sigprocmask()}\par
      \texttt{int sigprocmask(int how, const sigset\_t *restrict set,\\ 
              \tab\tab\tab sigset\_t *restrict oldset);}\\
      A seconda del valore di \textbf{how} e di \textbf{set}, la maschera dei segnali del processo viene 
      cambiata. Nello specifico:
      \begin{itemize}
        \item how = SIG\_BLOCK: i segnali in set sono aggiunti alla maschera;
        \item how = SIG\_UNBLOCK: i segnali in set sono rimossi dalla maschera;
        \item how = SIG\_SETMASK: set diventa la maschera.
      \end{itemize}
      Se oldset non \ace nullo, in esso verr\aca salvata la vecchia maschera 
      (anche se set \ace nullo).\\
      \begin{flushleft}
        Esempio 1: \\
        \texttt{\#include <signal.h> \\
        int main()\{\\
        \halftab sigset\_t mod,old;\\
        \halftab sigfillset(\&mod); // Add all signals to the blocked list\\
        \halftab sigemptyset(\&mod); // Remove all signals from blocked list\\
        \halftab sigaddset(\&mod,SIGALRM); // Add SIGALRM to blocked list\\
        \halftab sigismember(\&mod,SIGALRM); // is SIGALRM in blocked list?\\
        \halftab sigdelset(\&mod,SIGALRM); // Remove SIGALRM from blocked list\\
        \halftab // Update the current mask with the signals in ‘mod’\\
        \halftab sigprocmask(SIG\_BLOCK,\&mod,\&old);\\
        \}}
      \end{flushleft}
      \begin{flushleft}
        Esempio 2: \\
        \texttt{\#include <signal.h> <unistd.h> <stdio.h> \\
                sigset\_t mod, old;\\
                int i = 0;\\
                void myHandler(int signo)\{\\
                \halftab printf("signal received\n"); i++;
                \}\\
                int main()\{\\
                \halftab printf("my id = \%d\n",getpid());\\
                \halftab signal(SIGUSR1,myHandler);\\
                \halftab sigemptyset(\&mod); //Initialise set\\
                \halftab sigaddset(\&mod,SIGUSR1); \\
                \halftab while(1) if(i==1) sigprocmask(SIG\_BLOCK,\&mod,\&old);\\
                \}}
      \end{flushleft}
    \end{flushleft}
    \begin{flushleft}
      \textbf{Verificare pending signals: sigpending()}\par
      \texttt{int sigpending(sigset\_t *set);}\\
      \begin{flushleft}
        Esempio:\\
        \texttt{\#include <signal.h> <unistd.h> <stdio.h> <stdlib.h>\\
                sigset\_t mod,pen;\\
                void handler(int signo)\{ \\
                \halftab printf("SIGUSR1 received\n");\\
                \halftab sigpending(\&pen);\\
                \halftab if(!sigismember(\&pen,SIGUSR1)) \\
                \tab printf("SIGUSR1 not pending\n");\\
                \halftab exit(0);\\
                \}\\
                int main()\{\\
                \halftab signal(SIGUSR1,handler);\\
                \halftab sigemptyset(\&mod);\\
                \halftab sigaddset(\&mod,SIGUSR1);\\
                \halftab sigprocmask(SIG\_BLOCK,\&mod,NULL);\\
                \halftab kill(getpid(),SIGUSR1);\\ 
                \halftab // sent but it’s blocked...\\
                \halftab sigpending(\&pen); \\
                \halftab if(sigismember(\&pen,SIGUSR1))\\
                \tab printf("SIGUSR1 pending\n");\\
                \halftab sigprocmask(SIG\_UNBLOCK,\&mod,NULL);\\
                \halftab while(1);\\
                \}}  
      \end{flushleft}
    \end{flushleft}
  \end{flushleft}
  \begin{flushleft}
    \textbf{sigaction() system call}\par
    \texttt{int sigaction(int signum, const struct sigaction *restrict act,
            struct sigaction *restrict oldact);} \\
    \begin{flushleft}
      \texttt{struct sigaction \{\\
      \halftab void (*sa\_handler)(int);\\
      \halftab void (*sa\_sigaction)(int, siginfo\_t *, void *);\\
      \halftab sigset\_t sa\_mask; //Signals blocked during handler\\
      \halftab int sa\_flags; //modify behaviour of signal\\
      \halftab void (*sa\_restorer)(void); //Deprecated, not POSIX\\
      \};}
    \end{flushleft}
    Esempio: \\
    \begin{flushleft}
      \texttt{\#include <signal.h> <unistd.h> <stdlib.h> <stdio.h> \\
      void handler(int signo)\{\\
      \halftab printf("signal received\n");\\
      \}\\
      int main()\{ \\
      \halftab struct sigaction sa; //Define sigaction struct\\
      \halftab sa.sa\_handler = handler; //Assign handler to struct field\\
      \halftab sigemptyset(\&sa.sa\_mask); //Define an empty mask \\
      \halftab sigaction(SIGUSR1,\&sa,NULL);\\
      \halftab kill(getpid(),SIGUSR1);\\
      \}}
    \end{flushleft}
  \end{flushleft}
  \begin{flushleft}
    \textbf{Inviare un payload con un segnale}\\
    \texttt{int sigqueue(pid\_t pid, int sig, const union sigval value);}\\
    Invia un segnale sig al processo identificato da pid, accompagnato da un payload 
    value. Quest'ultimo \ace rappresentato con la seguente union:
    \begin{flushleft}
      \texttt{union sigval \{\\
      \halftab int sival\_int;\\
      \halftab void *sival\_ptr;\\
      \};}
    \end{flushleft}
    Essa può contenere un puntatore qualsiasi o un intero, che verr\aca ricevuto solo ed 
    esclusivamente se il processo destinatario utilizza sigaction con SA\_SIGINFO.
    \begin{flushleft}
      Ricezione: \\
      \texttt{\#include <signal.h> <unistd.h> <stdlib.h> <stdio.h> \\
              void handler(int signo, siginfo\_t * info, void * empty)\{\\
              \halftab printf("Received integer\n",info->si\_value.sival\_int);\\
              \}\\
              int main()\{\\
              \halftab struct sigaction sa;\\
              \halftab sa.sa\_sigaction = handler;\\
              \halftab sigemptyset(\&sa.sa\_mask); \\
              \halftab sa.sa\_flags $|$= SA\_SIGINFO; // Use sa\_sigaction\\
              \halftab sigaction(SIGUSR1,\&sa,NULL);\\
              \halftab while(1);\\
              \}}
    \end{flushleft}
    \begin{flushleft}
      Invio: \\
      \texttt{int main(int argc, char ** argv)\{\\
        \halftab union sigval value;\\
        \halftab value.sival\_int = atoi(argv[2]); \\
        \halftab sigqueue(atoi(argv[1],SIGUSR1,value));\\
        \halftab while(1);\\
        \}}
    \end{flushleft}
  \end{flushleft}
\end{flushleft}