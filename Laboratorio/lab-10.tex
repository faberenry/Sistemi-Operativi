\section{Lab-10: Thread}
\begin{flushleft}
  I \textbf{thread} sono singole sequenze di esecuzione all'interno di un processo, aventi alcune 
  delle propriet\aca dei processi. I threads non sono indipendenti tra loro e condividono il 
  codice, i dati e le risorse del sistema assegnate al processo di appartenenza. Come 
  ogni singolo processo, i threads hanno alcuni elementi indipendenti, come lo stack, il 
  PC ed i registri del sistema.\\
  La creazione di threads consente un parallelismo delle operazioni in maniera rapida e 
  semplificata. Context switch tra threads \ace rapido, cos\aci come la loro creazione e 
  terminazione. Inoltre, la comunicazione tra threads \ace molto veloce.\\
  \textbf{Per la compilazione \ace necessario aggiungere il flag -pthread, ad esempio:
  gcc -o program main.c -pthread}
  \begin{flushleft}
    \textbf{IN C}\\
    In C i thread corrispondono a delle funzioni eseguite in parallelo al codice principale. 
    Ogni thread \ace \textbf{identificato da un ID} e pu\aco essere gestito come un processo figlio, con 
    funzioni che attendono la sua terminazione. 
    Sebbene da un punto di vista l'esecuzione di diversi thread sia sempre parallela, a 
    causa delle politiche di scheduling delle CPU, l'esecuzione può essere parallela solo 
    se la CPU la supporta, ossia se dispone di più core.
  \end{flushleft}
  \subsection{Creazione}
  \begin{flushleft}
    \textbf{Creazione}\\
    \begin{flushleft}
      \texttt{int pthread\_create(\\
              pthread\_t *restrict thread, /* Thread ID */\\
              const pthread\_attr\_t *restrict attr, /* Attributes */\\
              void *(*start\_routine)(void *), /* Function to be executed */\\
              void *restrict arg /* Parameters to above function */\\
              );}
    \end{flushleft}
    Quando si crea un nuovo thread, \ace necessario fornire un puntatore pthread\_t, che 
    verrà riempito con il nuovo ID generato. attr consente di modificare il 
    comportamento dei thread, mentre start\_routine serve a definire quale funzione 
    deve essere eseguita dal thread. arg \ace un puntatore void che pu\aco essere utilizzato per 
    passare qualsiasi argomento sia richiesto. \\
    NB: void* è il tipo di variabile pi\acu grande e pu\aco quindi essere usato per puntare a qualsiasi struttura di dati
  \end{flushleft}
  \begin{flushleft}
    \textbf{Esempio}\\
    \texttt{\#include <stdio.h> <pthread.h> <unistd.h> \\
    void * my\_fun(void * param)\{ \\
    \halftab printf("This is a thread that received \%d\n", *(int *)param);\\
    \halftab return (void *)3;\\
    \} \\
    int main(void)\{ \\
    \halftab pthread\_t t\_id;\\
    \halftab int arg=10;\\
    // We need to cast the augment to a void *. We are passing the address of the variable!\\
    \halftab pthread\_create(\&t\_id, NULL, my\_fun, (void *)\&arg);\\
    \halftab printf("Executed thread with id \%ld\n",t\_id);\\
    \halftab sleep(3);\\
    \} }
    Le modifiche al valore di \texttt{arg} vengono viste anche dal thread 
  \end{flushleft}
  \begin{flushleft}
    All'interno del kernel: \\
    \begin{flushleft}
      \texttt{\#include <pthread.h> \\
      void * my\_fun(void * param)\{ \\
      \halftab while(1);\\
      \}\\
      int main(void)\{ \\
      \halftab pthread\_t t\_id;\\
      \halftab pthread\_create(\&t\_id, NULL, my\_fun, NULL);\\
      \halftab while(1);\\
      \} }
    \end{flushleft}
  \end{flushleft}
  \subsection{Terminazione}
  \begin{flushleft}
    \textbf{Terminazione}\\
    Un nuovo thread termina in uno dei seguenti modi:
    \begin{itemize}
      \item Chiamando la funzione void pthread\_exit(void * retval); dal thread, 
      specificando un puntatore da restituire.
      \item Effettuando un return dalla funzione associata al thread, specificando un valore 
      di ritorno.
      \item Cancellando il thread da un altro thread.
      \item Qualche thread chiama exit(), o il thread che esegue main() ritorna dallo 
      stesso, terminando così tutti i threads.
    \end{itemize}
  \end{flushleft}
  \subsection{Cancellazione}
  \begin{flushleft}
    \textbf{Cancellazione}\\
    \begin{flushleft}
      \texttt{int pthread\_cancel(pthread\_t thread);}
    \end{flushleft}
    Invia una richiesta di cancellazione al thread specificato, il quale reagirà (come e 
    quando) a seconda di due suoi attributi: cancel state e cancel type. 
    Il cancel state di un thread definisce se il thread deve terminare o meno quando una 
    richiesta di cancellazione viene ricevuta. Il cancel type di un thread definisce come il 
    thread deve terminare. 
    Sia lo stato che il tipo possono essere definiti alla creazione del thread o durante la 
    sua esecuzione, all’interno del thread stesso.
  \end{flushleft}
  \subsection{Cambiare il thread cancel state}
  \begin{flushleft}
    \textbf{Cambiare il thread cancel state}\\
    \begin{flushleft}
      \texttt{int pthread\_setcancelstate(int state, int *oldstate);}
    \end{flushleft}
    Modifica il cancel state del thread in esecuzione. Mentre oldstate viene riempito 
    con lo stato precedente, state può contenere una delle seguenti macro:
    \begin{itemize}
      \item PTHREAD\_CANCEL\_ENABLE: ogni richiesta di cancellazione viene gestita a 
      seconda del type del thread. Questa è la modalità default.
      \item PTHREAD\_CANCEL\_DISABLE: ogni richiesta di cancellazione aspetterà fino a 
      che il cancel state del thread non diventa PTHREAD\_CANCEL\_ENABLE.
    \end{itemize}
  \end{flushleft}
  \subsection{Cambiare il thread cancel type }
  \begin{flushleft}
    \begin{flushleft}
      \texttt{int pthread\_setcanceltype(int type, int *oldtype); }
    \end{flushleft}
    Modifica il cancel type del thread in esecuzione. Mentre oldtype viene riempito con 
    il type precedente, type può contenere una delle seguenti macro:
    \begin{itemize}
      \item PTHREAD\_CANCEL\_DEFERRED: la terminazione aspetta l’esecuzione di un 
      cancellation point. Questa è la modalità default.
      \item PTHREAD\_CANCEL\_ASYNCHRONOUS: la terminazione avviene appena la richiesta 
      viene ricevuta.
    \end{itemize}
    Cancellation points sono delle specifiche funzioni definite nella libreria pthread.h 
    (list). Di solito sono system calls.
    \begin{flushleft}
      \textbf{Esempio 1}\\
      \texttt{\#include <stdio.h> <pthread.h> <unistd.h> \\
      int i = 1;\\
      void * my\_fun(void * param)\{ \\
      \halftab if(i--) \\
      \halftab pthread\_setcancelstate(PTHREAD\_CANCEL\_DISABLE, NULL); //Change mode\\
      \halftab printf("Thread \%ld started\n",*(pthread\_t *)param); sleep(3);\\
      \halftab printf("Thread \%ld finished\n",*(pthread\_t *)param);\\
      \}\\
      int main(void)\{\\
      \halftab pthread\_t t\_id1, t\_id2;\\
      \halftab pthread\_create(\&t\_id1, NULL, my\_fun, (void *)\&t\_id1); sleep(1); //Create\\
      \halftab pthread\_cancel(t\_id1); //Cancel\\
      \halftab printf("Sent cancellation request for thread \%ld\n",t\_id1);\\
      \halftab pthread\_create(\&t\_id2, NULL, my\_fun, (void *)\&t\_id2); sleep(1); //Create\\
      \halftab pthread\_cancel(t\_id2); //Cancel\\
      \halftab printf("Sent cancellation request for thread \%ld\n",t\_id2);\\
      \halftab sleep(5); printf("Terminating program\n");\\
      \}  }
    \end{flushleft}
    \begin{flushleft}
      \textbf{Esempio 2}\\
      \begin{flushleft}
        \texttt{\#include <stdio.h> <pthread.h> <unistd.h> <string.h> \\
        int tmp = 0;\\
        void * my\_fun(void * param)\{ \\ 
        \halftab pthread\_setcanceltype(*(int *)param,NULL); // Change type\\
        \halftab for (long unsigned i = 0; i < (0x9FFF0000); i++); //just wait\\
        \halftab tmp++;\\
        \halftab open("/tmp/tmp",O\_RDONLY); //Cancellation point!\\
        \} \\
        int main(int argc, char ** argv)\{ //call program with ‘async’ or ‘defer’\\
        \halftab pthread\_t t\_id1; int arg;\\
        \halftab if(!strcmp(argv[1],"async")) arg = PTHREAD\_CANCEL\_ASYNCHRONOUS;\\
        \halftab else if(!strcmp(argv[1],"defer")) arg = PTHREAD\_CANCEL\_DEFERRED;\\
        \halftab pthread\_create(\&t\_id1, NULL, my\_fun, (void *)\&arg); sleep(1); //Create\\
        \halftab pthread\_cancel(t\_id1); sleep(5); //Cancel\\
        \halftab printf("Tmp \%d\n",tmp);\\
        \} }
      \end{flushleft}
    \end{flushleft}
  \end{flushleft}
  \subsection{Aspettare un thread}
  \begin{flushleft}
    \textbf{Aspettare un thread: join}\\
    Un processo (thread) che avvia un nuovo thread può aspettare la sua terminazione 
    mediante la funzione: 
    \begin{flushleft}
      \texttt{int pthread\_join(pthread\_t thread, void ** retval);}
    \end{flushleft}
    Questa funzione ritorna quando il thread identificato da thread termina, o subito 
    se il thread \ace già terminato. Se il valore di ritorno del thread non è nullo (parametro 
    di pthread\_exit() o di return), esso viene salvato nella variabile puntata da 
    retval. Se il thread era stato cancellato, retval è riempito con PTHREAD\_CANCELED.\\
    Solo se il thread \ace joinable può essere aspettato! \\
    Un thread può essere aspettato da al massimo un thread
  \end{flushleft}
  \subsection{Detach state di un thread}
  \begin{flushleft}
    I thread sono creati per impostazione predefinita nello stato joinable, che consente a 
    un altro thread di attendere la loro terminazione tramite il comando 
    pthread\_join(). I thread joinable non rilasciano le loro risorse alla terminazione, 
    ma quando un thread li aspetta (salvando lo stato di uscita, come i sottoprocessi), o 
    alla terminazione del processo stesso. Al contrario, i thread detached rilasciano le 
    loro risorse immediatamente al termine, ma non permettono ad altri processi di 
    aspettarli.
    NB: un thread detached non può diventare joinable durante la sua esecuzione, 
    mentre \ace possibile il contrario.
    \begin{flushleft}
      \textbf{Cambiare il detach state}
      \begin{flushleft}
        \texttt{int pthread\_detach(pthread\_t thread);}
      \end{flushleft}
      Questo comando può essere eseguito da un thread qualunque e cambia il detach state 
      di thread da joinable a detached. 
      \textbf{Ricorda che una volta cambiato lo stato non si pu\aco invertire}
    \end{flushleft}
  \end{flushleft}
  \begin{flushleft}
    \textbf{Esempi join 1}\\
    \texttt{\#include <stdio.h> <pthread.h> <unistd.h> \\
    void * my\_fun(void * param)\{ \\
    \halftab printf("Thread \%ld started\n",*(pthread\_t *)param); sleep(3);\\
    \halftab char * str = "Returned string";\\
    \halftab pthread\_exit((void *)str); //or ‘return (void *) str;’\\
    \}\\
    int main(void)\{ \\
    \halftab pthread\_t t\_id;\\
    \halftab void * retFromThread; //This must be a pointer to void!\\
    \halftab pthread\_create(\&t\_id, NULL, my\_fun, (void *)\&t\_id); //Create\\
    \halftab pthread\_join(t\_id,\&retFromThread); // wait thread\\
    \halftab // We must cast the returned value!\\
    \halftab printf("Thread \%ld returned '\%s'\n",t\_id,(char *)retFromThread);\\
    \} }
  \end{flushleft}
  \begin{flushleft}
    \textbf{Esempi join 2}\\
    \texttt{\#include <stdio.h> <pthread.h> <unistd.h> \\
    void * my\_fun(void *param)\{ \\
    \halftab printf("This is a thread that received \%d\n", *(int *)param);\\
    \halftab return (void *)3;\\
    \} \\
    int main(void)\{ \\
    \halftab pthread\_t t\_id;\\
    \halftab int arg=10, retval; \\
    \halftab pthread\_create(\&t\_id, NULL, my\_fun, (void *)\&arg);\\
    \halftab printf("Executed thread with id \%ld\n",t\_id);\\
    \halftab sleep(3);\\
    \halftab pthread\_join(t\_id, (void **)\&retval); //A pointer to a void pointer\\
    \halftab printf("retval=\%d\n", retval);\\
    \}}
  \end{flushleft}
  \begin{flushleft}
    \textbf{Esempio join 3}\\
    \texttt{\#include <stdio.h> <pthread.h> <unistd.h> \\
    void * my\_fun(void *param)\{\\
    \halftab printf("This is a thread that received \%d\n", *(int *)param);\\
    \halftab int i = *(int *)param * 2; //Local variable ceases to exist!\\
    \halftab return (void *)\&i;\\
    \}\\
    int main(void)\{\\
    \halftab pthread\_t t\_id;\\
    \halftab int arg=10, retval;\\ 
    \halftab pthread\_create(\&t\_id, NULL, my\_fun, (void *)\&arg);\\
    \halftab printf("Executed thread with id \%ld\n",t\_id);\\
    \halftab sleep(3);\\
    \halftab pthread\_join(t\_id, (void **)\&retval); //A pointer to a void pointer\\
    \halftab printf("retval=\%d\n", retval);\\
    \}}
  \end{flushleft}
  \subsection{Attributi thread}
  \begin{flushleft}
    Ogni thread viene creato con gli attributi specificati nella struttura 
    \texttt{pthread\_attr\_t}. Questa struttura, analogamente a quella utilizzata per gestire le 
    maschere di segnale, è un oggetto utilizzato solo quando viene creato un thread ed è 
    quindi indipendente da esso (se cambia, gli attributi del thread non cambiano).\\
    La struttura deve essere inizializzata, il che imposta tutti gli attributi al loro valore 
    predefinito. Una volta utilizzata e non più necessaria, la struct deve essere distrutta.\\
    I vari attributi della struttura possono e devono essere modificati individualmente 
    con alcune funzioni dedicate.
    \begin{itemize}
      \item \texttt{int pthread\_attr\_init(pthread\_attr\_t *attr)}
      Inizializza la struttura con tutti gli attributi default. 
      \item \texttt{int pthread\_attr\_destroy(pthread\_attr\_t *attr)}
      Distrugge la struttura.
      \item \texttt{int pthread\_attr\_setxxxx(pthread\_attr\_t *attr, params);}
      Imposta un certo attributo ad un certo valore.
      \item int pthread\_attr\_getxxxx(const pthread\_attr\_t *attr, params);
      Ottiene un certo attributo
    \end{itemize}
  \end{flushleft}
  \begin{flushleft}
    \textbf{Esempio attributi}\\
    \texttt{\#include <stdio.h> <pthread.h> <unistd.h> \\
    void * my\_fun(void *param)\{\\
    \halftab printf("This is a thread that received \%d\n", *(int *)param);return (void*)3;\\
    \}\\
    int main(void)\{ \\
    \halftab pthread\_t t\_id; pthread\_attr\_t attr;\\
    \halftab int arg=10, detachSTate;\\
    \halftab pthread\_attr\_init(\&attr);\\
    \halftab pthread\_attr\_setdetachstate(\&attr,PTHREAD\_CREATE\_DETACHED); //Set detached\\
    \halftab pthread\_attr\_getdetachstate(\&attr,\&detachSTate); //Get detach state\\
    \halftab if(detachSTate == PTHREAD\_CREATE\_DETACHED) printf("Detached\n"); \\
    \halftab pthread\_create(\&t\_id, \&attr, my\_fun, (void *)\&arg);\\
    \halftab printf("Executed thread with id \%ld\n",t\_id);\\
    \halftab pthread\_attr\_setdetachstate(\&attr,PTHREAD\_CREATE\_JOINABLE); //Inneffective\\
    \halftab sleep(3); pthread\_attr\_destroy(\&attr);\\
    \halftab int esito = pthread\_join(t\_id, (void **)\&detachSTate); \\
    \halftab printf("Esito '\%d' is different 0\n", esito);\\
    \} }
  \end{flushleft}
  \subsection{Thread e Segnali}
  \begin{flushleft}
    Quando viene inviato un segnale ad un processo non si può sapere quale thread 
    andrà a gestirlo. Per evitare problemi o comportamenti inattesi, è importante gestire 
    correttamente le maschere dei segnali dei singoli thread con \\ 
    \texttt{pthread\_attr\_setsigmask\_np(pthread\_attr\_t *attr,const sigset\_t *sigmask)} \\
    la quale usa *sigmask per impostare la maschera dei segnali nella struttura *attr.\\
    È poi possibile usare funzioni come \texttt{sigwait()} e \texttt{sigwaitinfo()} per gestire 
    l’attesa di un segnale. Infine, è possibile inviare un segnale ad un thread specifico usando \\ 
    \texttt{int pthread\_kill(pthread\_t thread, int sig);}
  \end{flushleft}
\end{flushleft}
\subsection{Mutex}
\begin{flushleft}
  \textbf{Problema della sincronizzazione}\\
  Quando eseguiamo un programma con più thread essi condividono alcune risorse, tra 
  le quali le variabili globali. Se entrambi i thread accedono ad una sezione di codice 
  condivisa ed hanno la necessità di accedervi in maniera esclusiva allora dobbiamo 
  instaurare una sincronizzazione. I risultati, altrimenti, potrebbero essere inaspettati.
  \begin{flushleft}
    \textbf{Esempio}\\
    \texttt{\#include <pthread.h> <stdlib.h> <unistd.h><stdio.h> \\
    pthread\_t tid[2]; \\
    int counter = 0;\\
    void *thr1(void *arg)\{\\
    \halftab counter = 1;\\
    \halftab printf("Thread 1 has started with counter \%d\n",counter);\\
    \halftab for (long unsigned i = 0; i < (0x00FF0000); i++); //wait some cycles\\
    \halftab counter += 1;\\
    \halftab printf("Thread 1 expects 2 and has: \%d\n", counter);\\
    \} \\
    void *thr2(void *arg)\{\\
    \halftab counter = 10;\\
    \halftab printf("Thread 2 has started with counter \%d\n",counter);\\
    \halftab for (long unsigned i = 0; i < (0xFFF0000); i++); //wait some cycles\\
    \halftab counter += 1;\\
    \halftab printf("Thread 2 expects 11 and has: \%d\n", counter);\\
    \}\\
    void main(void)\{ \\
    \halftab pthread\_create(\&(tid[0]), NULL, thr1,NULL);\\
    \halftab pthread\_create(\&(tid[1]), NULL, thr2,NULL);\\
    \halftab pthread\_join(tid[0], NULL);\\
    \halftab pthread\_join(tid[1], NULL);\\
    \}}
  \end{flushleft}
  I mutex sono dei semafori imposti ai thread. Essi possono proteggere una 
  determinata sezione di codice, consentendo ad un thread di accedervi in maniera 
  esclusiva fino allo sblocco del semaforo. Ogni thread che vorrà accedere alla stessa 
  sezione di codice dovrà aspettare che il semaforo sia sbloccato, andando in sleep fino 
  alla sua prossima schedulazione.\\
  I mutex vanno inizializzati e poi assegnati ad una determinata sezione di codice. Il 
  blocco e sblocco è manuale.\\
  NB: i mutex non regolano l’accesso alla memoria o alle variabile, ma solo a porzioni 
  di codice.
  \subsection{Creazione Eliminazione Mutex}
  \begin{flushleft}
    \texttt{int pthread\_mutex\_init(pthread\_mutex\_t *restrict mutex, const pthread\_mutexattr\_t *restrict attr)\\
    int pthread\_mutex\_destroy(pthread\_mutex\_t *mutex)}
  \end{flushleft}
  \subsection{Bloccaggio e sbloccaggio}
  \begin{flushleft}
    \texttt{int pthread\_mutex\_lock(pthread\_mutex\_t *mutex)\\
    int pthread\_mutex\_unlock(pthread\_mutex\_t *mutex)}
    \begin{flushleft}
      Dopo essere stato creato, un mutex deve essere bloccato per essere efficace. Non 
      appena un thread lo blocca, un altro thread deve attendere che venga sbloccato 
      prima di procedere al suo blocco.\\
      Quando si richiama il blocco, un thread attende che il mutex sia libero e poi lo blocca
    \end{flushleft}
  \end{flushleft}
  \begin{flushleft}
    \textbf{Esempio}\\
    \texttt{\#include <pthread.h> <stdlib.h> <unistd.h><stdio.h> \\
    pthread\_mutex\_t lock;\\
    pthread\_t tid[2];\\
    int counter = 0;\\
    void* thr1(void* arg)\{\\
    \halftab pthread\_mutex\_lock(\&lock);\\
    \halftab counter = 1;\\
    \halftab printf("Thread 1 has started with counter \%d\n",counter);\\
    \halftab for (long unsigned i = 0; i < (0x00FF0000); i++);\\
    \halftab counter += 1;\\
    \halftab pthread\_mutex\_unlock(\&lock);\\
    \halftab printf("Thread 1 expects 2 and has: \%d\n", counter);\\
    \} \\
    void * thr2(void* arg)\{ \\
    \halftab pthread\_mutex\_lock(\&lock);\\
    \halftab counter = 10;\\
    \halftab printf("Thread 2 has started with counter \%d\n",counter);\\
    \halftab for (long unsigned i = 0; i < (0xFFF0000); i++);\\
    \halftab counter += 1;\\
    \halftab pthread\_mutex\_unlock(\&lock);\\
    \halftab printf("Thread 2 expects 11 and has: \%d\n", counter);\\
    \} \\
    int main(void)\{\\
    \halftab pthread\_mutex\_init(\&lock, NULL);\\
    \halftab pthread\_create(\&(tid[0]), NULL, thr1,NULL);\\
    \halftab pthread\_create(\&(tid[1]), NULL, thr2,NULL);\\
    \halftab pthread\_join(tid[0], NULL);\\
    \halftab pthread\_join(tid[1], NULL);\\
    \halftab pthread\_mutex\_destroy(\&lock);\\
    \} }
  \end{flushleft}
\end{flushleft}
\begin{flushleft}
  I “thread” sono una sorta di “processi leggeri” che permettono di eseguire funzioni 
  “in concorrenza” in modo più semplice rispetto alla generazioni di processi veri e 
  propri (forking).\\
  I “MUTEX” sono un metodo semplice ma efficace per eseguire sezioni critiche in 
  processi multithread.\\
  È importante limitare al massimo la sezione critica utilizzando lock/unlock per la 
  porzione di codice più piccola possibile, e solo se assolutamente necessario (per 
  esempio quando possono capitare accessi concorrenti).
\end{flushleft}
