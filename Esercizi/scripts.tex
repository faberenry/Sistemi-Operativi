\section{Script}
\subsection{lez-01}
\begin{flushleft}
  1.0) Esempio prova \\
  \texttt{
  nargs=\$\# \\
  while [[ \$1 != "" ]]; do \\
    \tab echo "ARG=\$1" \\
    \tab shift \\
  done}\\
\end{flushleft}

\begin{flushleft}
  1.1) Scrivere uno script che dato un qualunque numero di argomenti li restituisca in 
  output in ordine inverso.
  
  \texttt{\\
  lista=()\\
  while [[ \$1 != "" ]]; do \\
   \tab lista=( "\$1" "\${lista[@]}" )\\
   \tab shift\\
  done \\
  echo "\${lista[@]}"}\\
\end{flushleft}

\begin{flushleft}
  1.2) Scrivere uno script che mostri il contenuto della cartella corrente in ordine 
  inverso rispetto all'output generato da "ls" (che si può usare ma senza opzioni). 
  Per semplicit\aca, assumere che tutti i file e le cartelle non abbiano spazi nel nome.
  
  \texttt{\\dati=(\$( ls ))\\
  lista=()\\
  for i in \${!dati[@]}; do \\
    \tab lista=( "\${dati[\$i]}" "\${lista[@]}" )\\
  done \\
  echo \$( ls )\\
  echo "\${lista[@]}"}
\end{flushleft}

\subsection{lez-02}

\begin{flushleft}
  2.1) Creare un makefile con una regola help di default che mostri una nota 
  informativa, una regola backup che crei un backup di una cartella appendendo 
  “.bak” al nome e una restore che ripristini il contenuto originale.
  
  \texttt{SHELL := /bin/bash\\
  FOLDER := /tmp\\
  \tab \\
  help: \\
  \tab	@echo "make -f makefile\_backup backup FOLDER=<path>"\\
  \tab \\
  backup:\\
  	\tab @echo "Backup of folder \$(FOLDER) as \$(FOLDER).bak..." ; \\
    \tab sleep 2s\\
  	\tab @[[ -d \$(FOLDER).bak ]] \&\& echo "?Error" || cp -rp \$(FOLDER) \$(FOLDER).bak\\
    \tab \\
  restore: \$(FOLDER).bak\\
  	\tab @echo "Restore of folder  \$(FOLDER) from \$(FOLDER).bak..." ; \\
    \tab sleep 2s\\
  	\tab @[[ -d \$(FOLDER) ]] \&\& echo "?Error" || cp -rp \$(FOLDER).bak \$(FOLDER)\\
  \tab \\
  .PHONY: help backup restore\\
  } 
\end{flushleft}
\subsection{lez-03}
\begin{flushleft}
3.1) Scrivere un'applicazione che data una stringa come argomento ne stampa a video la 
lunghezza, ad esempio:
\texttt{./lengthof "Questa frase ha 28 caratteri"}\\
deve restituire a video il numero 28 \\ 
\texttt{\#include <stdio.h>\\
int main(int argc, char **argv)\{ \\
  \tab int code = 0; \\
  \tab int len = 0; \\
  \tab char *p; \\
  \tab printf("\%d\textbackslash n", argc); \\
  \tab if(argc != 2) \{ \\
    \tab \tab fprintf(stderr,"Usage: \%s <stringa>\textbackslash n", argv[0]); \\
    \tab \tab code = 2; \\
  \tab \}else \{ \\
    \tab \tab p = argv[1]; //Copy pointer to first argument \\
    \tab \tab while (*p != 0 )\{  //Check if character is termination\\
      \tab \tab \tab p++; //Move to next char \\
      \tab \tab \tab len++; //Increase length count \\
    \tab \tab \} \\
    \tab \tab printf("\%d\textbackslash n", len); \\
  \tab \} \\
  \tab return code; \\
\}
}
\end{flushleft}

\begin{flushleft}
  3.2) Scrivere un'applicazione che definisce una lista di argomenti validi e legge quelli passati 
  alla chiamata verificandoli e memorizzando le opzioni corrette, restituendo un errore in 
  caso di un'opzione non valida. \\
  \textit{Non ho voglia di copiarlo}
\end{flushleft}

\begin{flushleft}
  3.3) Realizzare funzioni per stringhe \texttt{char *stringrev(char * str)} (inverte ordine 
caratteri) e \texttt{int stringpos(char * str, char chr)} (cerca chr in str e restituisce la 
posizione) \\
(Non completo, controllare stringrev, da segmentation fault)
\end{flushleft}

\subsection{lez-04}