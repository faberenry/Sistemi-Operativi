\section{Lab-07}
\begin{flushleft}
  \textbf{Errori in C} \par 
  Durante l'esecuzione di un programma ci possono essere diversi tipi di errori: system 
  calls che falliscono, divisioni per zero, problemi di memoria \dots \par
  Alcuni di questi errori non fatali, come una system call che fallisce, possono essere 
  indagati attraverso la variabile \texttt{errno}. Questa variabile globale contiene l’ultimo 
  codice di errore generato dal sistema. \par 
  Per convertire il codice di errore in una stringa comprensibile si pu\aco usare la 
  funzione \texttt{char *strerror(int errnum)}. \\
  In alternativa, la funzione \texttt{void perror(const char *str)} che stampa su 
  \textbf{stderr} la stringa passatagli come argomento concatenata, tramite ': ', con 
  strerror(errno). 
  \begin{flushleft}
    Esempio: \par 
    \texttt{\#include <stdio.h> <errno.h> <string.h> \\
            extern int errno; // declare external global variable\\
            int main(void)\{\\
            \halftab FILE * pf;\\
            \halftab pf = fopen ("nonExistingFile.boh", "rb"); //Try to open file\\
            \halftab if (pf == NULL) \{ //something went wrong!\\
            \tab fprintf(stderr, "errno = \%d\n", errno); \\
            \tab perror("Error printed by perror");\\
            \tab fprintf(stderr,"Strerror: \%s\n", strerror(errno));\\
            \halftab \} else \{\\
            \tab fclose (pf);\\
            \halftab \} \\
            \}}
  \end{flushleft}
  \begin{flushleft}
    Esempio: \\
    \texttt{\#include <stdio.h> <errno.h> <string.h> <signal.h>\\
            extern int errno; // declare external global variable\\
            int main(void)\{ \\
            \halftab int sys = kill(3443,SIGUSR1); //Send signal to non exist proc\\
            \halftab if (sys == -1) \{ //something went wrong!\\
            \tab fprintf(stderr, "errno = \%d\n", errno); \\
            \tab perror("Error printed by perror");\\
            \tab fprintf(stderr,"Strerror: \%s\n", strerror(errno));\\
            \halftab \} else \{\\
            \tab printf("Signal sent\n");\\
            \halftab \}\\
            \}}
  \end{flushleft}
\end{flushleft}
\subsection{Gruppi}
\begin{flushleft}
  \textbf{Process groups} \par 
  All'interno di Unix i processi vengono raggruppati secondi vari criteri, dando vita a 
  sessioni, gruppi e threads\\
  \textbf{Perch\ace i gruppi:} I process groups consentono una migliore gestione dei segnali e della comunicazione 
  tra i processi.\\
  Un processo, per l'appunto, pu\aco:
  \begin{itemize}
    \item Aspettare che tutti i processi figli appartenenti ad un determinato gruppo
    terminino;\par
    \texttt{waitpid(-33,NULL,0); // Wait for a children in group 33}
    \item Mandare un segnale a tutti i processi appartenenti ad un determinato gruppo\\
    \texttt{kill(-45,SIGTERM); // Send SIGTERM to all children in grp 45}
  \end{itemize}
  L'organizzazione dei processi in gruppi consente di organizzare meglio le 
  comunicazione e di coordinare le operazioni avendo in particolare la possibilit\aca di 
  inviare dei segnali ai gruppi nel loro complesso.
  \begin{flushleft}
    \textbf{Gruppi in Unix}\\
    Mentre, generalmente, una sessione \ace collegata ad un terminale, i processi vengono 
    raggruppati nel seguente modo:
    \begin{itemize}
      \item In bash, processi concatenati tramite pipes appartengono allo stesso gruppo: \\
            \texttt{cat /tmp/ciao.txt | wc -l | grep '2'}
      \item Alla loro creazione, i figli di un processo ereditano il gruppo del padre
      \item Inizialmente, tutti i processi appartengono al gruppo di 'init', ed ogni processo 
            pu\aco cambiare il suo gruppo in qualunque momento.  
    \end{itemize}
    Il processo il cui PID \ace uguale al proprio \textbf{GID} \ace detto \textbf{process group leader}
  \end{flushleft}
  \begin{flushleft}
    \textbf{Group system calls}\\
    \begin{flushleft}
      \texttt{int setpgid(pid\_t pid, pid\_t pgid); //set GID of proc. (0=self)\\
              pid\_t getpgid(pid\_t pid); // get GID of process (0=self) }      
    \end{flushleft}
    \begin{flushleft}
      Esempio: \\
      \texttt{\#include <stdio.h> <unistd.h> <sys/wait.h> \\
      int main(void)\{ \\
      \halftab int isChild = !fork(); //new child\\
      \halftab printf("PID \%d PPID: \%d GID \%d\n", \\
                      \tab \tab getpid(),getppid(),getpgid(0));\\
      \halftab if(isChild)\{ \\
      \halftab isChild = !fork(); //new child\\
      \halftab if(!isChild) setpgid(0,0); // Become group leader\\
      \tab sleep(1);\\
      \tab fork(); //new child\\
      \tab printf("PID \%d PPID: \%d GID \%d\n",\\
                  \halftab \tab \tab getpid(),getppid(),getpgid(0));\\
      \halftab \}; \\ 
      \halftab while(wait(NULL)>0);\\
      \}}
    \end{flushleft}
    \subsection{Segnnali ai gruppi}
    \begin{flushleft}
      \textbf{Mandare segnali ai gruppi} \par 
      Nel prossimo esempio:
      \begin{enumerate}
        \item Processo ‘ancestor’ crea un figlio
        \begin{enumerate}
          \item Il figlio cambia il proprio gruppo e genera 3 figli (Gruppo1)
          \item I 4 processi aspettano fino all'arrivo di un segnale
        \end{enumerate}
        \item Processo ‘ancestor’ crea un secondo figlio
        \begin{enumerate}
          \item Il figlio cambia il proprio gruppo e genera 3 figli (Gruppo2)
          \item I 4 processi aspettano fino all’arrivo di un segnale
        \end{enumerate}
        \item Processo ‘ancestor’ manda due segnali diversi ai due gruppi
      \end{enumerate}
      \texttt{\#include <stdio.h><unistd.h><sys/wait.h><signal.h><stdlib.h>\\
      void handler(int signo)\{\\
      \halftab printf("[\%d,\%d] sig\%d received\n",getpid(),getpgid(0),signo); \\
      \halftab sleep(1); exit(0);\\
      \}\\
      int main(void)\{ \\
      \halftab signal(SIGUSR1,handler);\\
      \halftab signal(SIGUSR2,handler);\\
      \halftab int ancestor = getpid(); int group1 = fork(); int group2; \\
      \halftab if(getpid()!=ancestor )\{ // First child \\
      \tab setpgid(0,getpid()); // Become group leader\\
      \tab fork();fork(); //Generated 3 children in new group\\
      \halftab \}else \{ \\
      \tab group2 = fork();\\
      \tab if(getpid()!=ancestor)\{ // Second child\\
      \tab \halftab setpgid(0,getpid()); // Become group leader\\
      \tab \halftab fork();fork();\\ 
      \tab \}\\
      \halftab \} //Generated 3 children in new group\\
      \halftab if(getpid()==ancestor)\{\\
      \tab printf("[\%d]Ancestor and I'll send signals\n",getpid());\\
      \tab sleep(1);\\
      \tab kill(-group2,SIGUSR2); //Send SIGUSR2 to group2\\
      \tab kill(-group1,SIGUSR1); //Send SIGUSR1 to group1\\
      \halftab \}else\{\\
      \tab printf("[\%d,\%d]chld waiting signal\n", getpid(),getpgid(0));\\
      \tab while(1);\\
      \halftab \} \\
      \halftab while(wait(NULL)>0);\\
      \halftab printf("All children terminated\n");\\
      \}}
    \end{flushleft}
    \begin{flushleft}
      \textbf{Wait figli in un gruppo}\par 
      Nel prossimo esempio:
      \begin{enumerate}
        \item 1. Processo ‘ancestor’ crea un figlio
        \begin{enumerate}
          \item Il figlio cambia il proprio gruppo e genera 3 figli (Gruppo1)
          \item I 4 processi aspettano 2 secondi e terminano
        \end{enumerate}
        \item Processo ‘ancestor’ crea un secondo figlio
        \begin{enumerate}
          \item Il figlio cambia il proprio gruppo e genera 3 figli (Gruppo2)
          \item I 4 processi aspettano 4 secondi e terminano
        \end{enumerate}
        \item Processo ‘ancestor’ aspetta la terminazione dei figli del gruppo1
        \item Processo ‘ancestor’ aspetta la terminazione dei figli del gruppo2
      \end{enumerate}
      \texttt{\#include <stdio.h><unistd.h><sys/wait.h> \\
      int main(void)\{\\
      \halftab int group1 = fork(); \\
      \halftab int group2;\\
      \halftab if(group1 == 0)\{ // First child\\
      \tab setpgid(0,getpid()); // Become group leader\\
      \tab fork(); fork(); //Generated 4 children in new group\\
      \tab sleep(2); return; //Wait 2 sec and exit\\
      \halftab \}else\{\\
      \tab group2 = fork();\\
      \tab if(group2 == 0)\{\\
      \tab \halftab setpgid(0,getpid()); // Become group leader\\
      \tab \halftab fork(); fork(); //Generated 4 children\\
      \tab \halftab sleep(4); return; //Wait 4 sec and exit\\
      \tab \}\\
      \halftab \}\\
      \halftab sleep(1); //make sure the children changed their group\\
      \halftab while(waitpid(-group1,NULL,0)>0);\\
      \halftab printf("Children in \%d terminated\n",group1);\\
      \halftab while(waitpid(-group2,NULL,0)>0);\\
      \halftab printf("Children in \%d terminated\n",group2);\\
      \} \\
      }
    \end{flushleft}
  \end{flushleft}
\end{flushleft}