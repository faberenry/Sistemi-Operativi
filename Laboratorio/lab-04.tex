\section{Lab-04: File}
\begin{flushleft}
  \textbf{File}\par 
  In unix ci sono due modi per interagre con i file:
  \begin{itemize}
    \subsection{Streams}
    \item \textbf{Streams}:forniscono strumenti come la formattazione dei dati, bufferizzazione, ecc… \\
          Utilizzando gli streams, un file \ace descritto da un puntatore a una struttura di tipo 
          FILE (definita in stdio.h). I dati possono essere letti e scritti in vari modi (un 
          carattere alla volta, una linea alla volta, ecc.) ed essere interpretati di conseguenza. \\
          \textbf{Manipolazione con streams} \par 
          \texttt{FILE *fopen(const char *filename, const char *mode)} \\
          Restituisce un FILE pointer per gestire il "filename" nella modalit\aca specificata:
          \begin{itemize}
            \item r: read
            \item w: write
            \item r+ read adn write 
            \item w+: read and write, create or overwrite
            \item a: write at the end
            \item a+: read and write at the end
          \end{itemize}
          \texttt{int fclose(FILE *stream)}: chiusura del file che ritorna un intero con lo stato della chiusura \\
          \textbf{Lettura del file}:
          \begin{itemize}
            \item \texttt{int fgetc(FILE *stream)}: restituisce un carattere dallo stream
            \item \texttt{char *fgets(char *str, int n, FILE *strem)}: ritorna una stringa da stream e la salva in 
                    str, si ferma quando sono stati letti n-1 caratteri, o viene letto un "\textbackslash n", oppure raggiunta la fine\\
                    Inserisce il carattere di terminazione. 
            \item \texttt{int fscanf(FILE *stream, const char *format)}: legge da stream dei dati, salvando 
                    ogni dato nelle variabili fornite, seguendo la stringa format 
            \item \texttt{int feof(FILE *strem)}: restituisce 1 se lo stream \ace arrivato alla fine del file.
          \end{itemize}
          \textbf{Scrittura del file}:
          \begin{itemize}
            \item \texttt{int fputc(int chr, FILE *stream)}: scrive un singolo carattere char  su stream 
            \item \texttt{int fputs(const char *str, FILE *stream)}: scrive una stringa str su file 
            \item \texttt{int fprintf(FILE *stream, const char *format)}:scrive il contenuto di alcune variabili su stream 
                    seguendo la stringa format
          \end{itemize}
          \textbf{Flush e Rewind} \par 
          Seguendo l'immagine, il contenuto di un file viene letto e scritto con degli streams 
          (dei buffer) di dati. Come tali, \ace comprensibile come queste operazioni non siano 
          immediate: i dati vengono scritti sul buffer e solo successivamente scritti sul file. Il 
          \textbf{flush} \ace l'operazione che trascrive il file dallo stream. Questa operazione avviene 
          quando: 
          \begin{itemize}
            \item Il programma termina con un return dal main o con exit().
            \item \texttt{fprintf()} inserisce una nuova riga.
            \item \texttt{int fflush(FILE *stream)} viene invocato.
            \item \texttt{void rewind(FILE *stream)} viene invocato.
            \item \texttt{fclose()} viene invocato
          \end{itemize}
          \textbf{rewind} consente inoltre di ripristinare la posizione della testina all'inizio del file. \\
          \begin{flushleft}
            \textbf{Esempio 1}:\\
            \texttt{\#include <stdio.h>\\
            FILE *ptr; //Declare stream file\\
            ptr = fopen("filename.txt","r+"); //Open\\
            int id;\\
            char str1[10], str2[10];\\
            while (!feof(ptr))\{ //Check end of file\\
            \tab //Read int, word and word \\
            \tab fscanf(ptr,"\%d \%s \%s", \&id, str1, str2);\\
            \tab printf("\%d \%s \%s\textbackslash n",id,str1,str2); \\
            \}\\
            printf("End of file\textbackslash n");\\
            fclose(ptr); //Close file\\
            }\par 
            \textbf{Esempio 2}:\\
            \texttt{\#include <stdio.h>\\
              \#define N 10\\
              FILE *ptr;\\
              ptr = fopen("fileToWrite.txt","w+"); \\
              fprintf(ptr,"Content to write"); //Write content to file\\
              rewind(ptr); // Reset pointer to begin of file\\
              char chAr[N], inC;\\
              fgets(chAr,N,ptr); // store the next N-1 chars from ptr in chAr\\
              printf("’\%d’ ‘\%s’",chAr[N-1], chAr);\\
              do\{\\
              \tab inC = fgetc(ptr); // return next available char or EOF\\
              \tab printf("\%c",inC);\\
              \}while(inC != EOF); \\
              printf("\textbackslash n");\\
              fclose(ptr);\\
            }
          \end{flushleft}
    \subsection{File Descriptor}
    \item \textbf{File descriptor} interfaccia di basso livello costituita dalle system call messe a 
            disposizione dal kernel. Un file \ace descritto da un semplice intero (\textbf{file descriptor}) che punta alla rispettiva 
            entry nella file table del sistema operativo. I dati possono essere letti e scritti 
            soltanto un buffer alla volta di cui spetta al programmatore stabilire la dimensione.\\
            Un insieme di system call permette di effettuare le operazioni di input e output 
            mantenendo un controllo maggiore su quanto sta accadendo a prezzo di 
            un'interfaccia meno amichevole.\par 
            Per accedere al contenuto di un file bisogna creare un canale di comunicazione con il 
            kernel, aprendo il file con la system call open la quale localizza l’i-node del file e 
            aggiorna la file table del processo.\\
            A ogni processo \ace associata una tabella dei file aperti 
            di dimensione limitata, dove ogni elemento della 
            tabella rappresenta un file aperto dal processo ed è 
            individuato da un indice intero (il "file descriptor")\\
            I file descriptor 0, 1 e 2 individuano normalmente 
            standard input (0=stdin), output (1=stdout) ed error (2=stderr) (aperti automaticamente). \\
            Il kernel gestisce l'accesso ai files attraverso due strutture dati: 
            \begin{itemize}
              \item La \textbf{la tabella dei files attivi} che contiene una copia dell'i-node di ogni file 
                    aperto (per efficienza)
              \item La \textbf{tabella dei files aperti} che contiene un elemento per ogni file aperto e 
                    non ancora chiuso\\ Questo elemento contiene:
                      \begin{itemize}
                        \item I/O pointer: posizione corrente nel file
                        \item i-node pointer: Puntatore a inode corrispondente
                      \end{itemize}
                    La tabella dei file aperti pu\aco avere pi\acu elementi corrispondenti allo stesso file
            \end{itemize} 
          \textbf{Apertura e chiusura file} \\
          \texttt{int open(const char *pathname, int flags[, mode\_t mode]);}\par 
          \textbf{flags}: interi che definiscono l'apertura del file:
          \begin{itemize}
            \item \textbf{O\_RDONLY, O\_WRONLY, O\_RDWR}: almeno uno \ace obbligatorio
            \item \textbf{O\_CREAT, O\_EXCL}: crea il file se non esiste, come creat per\aco fallisce se il file esiste gi\aca
            \item \textbf{O\_APPEND}: apre il file in append
            \item \textbf{O\_TRUNC}: cancella il contenuto del file (se usato in scrittura)
          \end{itemize}
          \textbf{mode}: interi per i privilegi da assegnare al nuovo file, \textbf{ S\_IRUSR, S\_IWUSR, S\_IXUSR, S\_IRWXU, S\_IRGRP, …, S\_IROTH}\par 
          \texttt{int close(int fd)}\par 
          \textbf{Lettura e scrittura file}\\
          \begin{itemize}
            \item \texttt{ssize\_t read (int fd, void *buf, size\_t count);}\\ legge dal file e salva nel buffer buf, 
                  fino a count bytes di dati dal file associato con il file descriptor fd
            \item \texttt{ssize\_t write(int fd, const void *buf, size\_t count);}\\ scrive sul file associato al file descriptor fd fino a count bytes di dati 
                  dal buffer 'buf'
            \item \texttt{off\_t lseek(int fd, off\_t offset, int whence);}\\ riposiziona l'offset del file a seconda dell'argomento 
                    offset partendo da una certa posizione whence. Tre posizioni fondamentali: \textbf{SEEK\_SET} (inizio file), \textbf{SEEK\_CUR} (posizione attuale), 
                    \textbf{SEEk\_END} (fine file)
          \end{itemize}          
          \begin{flushleft}
            \textbf{Esempio 1}: \\
            \texttt{\#include <unistd.h> \\
            \#include <stdio.h>\\
            \#include <fcntl.h>\\
            //Open new file in Read only\\
            int openedFile = open("filename.txt", O\_RDONLY);\\
            char content[10]; \\
            int canRead;\\
            do\{\\
            \tab //Read 9B to content\\
            \tab bytesRead = read(openedFile,content,9);\\
            \tab content[bytesRead]=0;\\
            \tab printf("\%s",content);\\
            \} while(bytesRead > 0);\\
            close(openedFile); }
          \end{flushleft}
          \begin{flushleft}
            \textbf{Esempio 2}:\\
            \texttt{\#include <unistd.h>\\
            \#include <fcntl.h>\\
            \#include <string.h>\\
            //Open file (create it with user R and W permissions)\\
            int openFile = open("name.txt", O\_RDWR | O\_CREAT, S\_IRUSR | S\_IWUSR);\\
            char toWrite[] = "Professor";\\
            write(openFile, "hello world\textbackslash n", strlen("hello world\textbackslash n")); //Write to file\\
            lseek(openFile, 6, SEEK\_SET); // move I/O pointer\\
            write(openFile, toWrite, strlen(toWrite)); //Write to file\\
            close(openFile); } 
          \end{flushleft}
  \end{itemize}
  \subsection{Canali standard}
  \textbf{Canali standard}
  \begin{itemize}
    \item I canali standard (in/err/out), sono rappresentati con strutture stream (stdin, stderr, stdout) e macro 
          (STDIN\_FILENO, STDERR\_FILENO, STDOUT\_FILENO)
    \item La funzione \texttt{fileno} restituisce l'indice di uno stream per cui:
          \begin{itemize}
            \item \texttt{fileno(stdin) = STDIN\_FILENO} ( = 0)
            \item \texttt{fileno(stdout) = STDOUT\_FILENO} (= 1)
            \item \texttt{fileno(stderr) = STDERR\_FILENO} (= 2)
          \end{itemize}
    \item \texttt{isatty(stdin) == 1} con esecuzione interattiva, 0 altrimenti 
    \item \texttt{printf("ciao") $\equiv$ fprintf(stdout, "ciao")}
    \item \textbf{Esempio} \\
          \begin{flushleft}
            \texttt{\#include <stdio.h>\\
            \#include <unistd.h>\\
            void main() \{ \\
             \tab printf("stdin: stdin ->\_flags = \%hd, STDIN\_FILENO = \\
             \tab \halftab \%d\textbackslash n", stdin->\_flags, STDIN\_FILENO\\
             \tab ); \\
             \tab printf("stdout: stdout->\_flags = \%hd, STDOUT\_FILENO =\\
             \tab \halftab \%d\textbackslash n",stdout->\_flags, STDOUT\_FILENO \\
             \tab ); \\
             \tab printf("stderr: stderr->\_flags = \%hd, STDOUT\_FILENO = \\
             \tab \halftab \%d\textbackslash n", stderr->\_flags, STDERR\_FILENO\\
             \tab ); \\
             \} }
          \end{flushleft}
  \end{itemize}
  \subsection{Piping bash}
  \textbf{Piping con bash}
  \begin{flushleft}
    Normalmente un'applicazione eseguita da ash ha accesso ai canali standard, se le applicazioni sono usate 
    in un'operazione di piping (\texttt{ls | wc -l}) allora l'output  dell'applicazione 
    sulla sinistra diventa l'input dell'applicazione sulla destra e verranno eseguite \textbf{parallelamente} \par 
    \textbf{Esempio}: \\
    \begin{flushleft}
      \texttt{gcc src.c -o pip.out \\
              echo "hi how are you" | ./pip.out}\\
      \textit{src.c} \\
      \texttt{\#define MAXBUF 10\\
      \#include <stdio.h>\\
      \#include <string.h>\\
      int main() \{ \\
       \halftab char buf[MAXBUF]; \\
       \halftab fgets(buf, sizeof(buf), stdin); // may truncate!\\
       \halftab printf("\%s\textbackslash n", buf); \\
       \halftab return 0; \\
      \} }
    \end{flushleft}
    \begin{flushleft}
      \texttt{ls /tmp | ./inv.out} \par 
      \textit{Code}: \\
      \texttt{\#include <stdio.h>\\
      int main() \{ \\
       \halftab int c, d; \\
       \halftab // loop into stdin until EOF (as CTRL+D) \\
       \halftab // read from stdin \\
       \halftab while ((c = getchar()) != EOF) \{ \\
       \tab d = c; \\
       \tab if (c >= 'a' \&\& c <= 'z') d -= 32; \\
       \tab if (c >= 'A' \&\& c <= 'Z') d += 32; \\
       \tab putchar(d); // write to stdout \\
       \halftab \}; \\
       \halftab return (0); \\
      \} }
    \end{flushleft}
  \end{flushleft}
\end{flushleft}