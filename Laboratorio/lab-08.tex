\section{Lab-08: Pipe}
\subsection{Anonime}
\begin{flushleft}
  \textbf{Pipe anonime}
  \begin{flushleft}
    \textbf{Piping}\par 
    Il piping connette l'output (stdout e stderr) di un comando all'input (stdin) di un 
    altro comando, consentendo dunque la comunicazione tra i due. \\
    \begin{flushleft}
      Esempio: \\
      \texttt{ls . | sort -R \#stout $\rightarrow$ stdin\\
      ls nonExistingDir |\& wc \#stdout e stderr $\rightarrow$ stdin\\
      cat /etc/passwd | wc | less \#out $\rightarrow$ in, out$\rightarrow$ in}
    \end{flushleft}
    I processi sono eseguiti in concorrenza utilizzando un buffer: 
    \begin{itemize}
      \item Se pieno lo scrittore (left) si sospende fino ad avere spazio libero
      \item Se vuoto il lettore si sospende fino ad avere i dati
    \end{itemize}
    Esempio:
    \begin{flushleft}
      \texttt{./output.out | ./input.out}\\
      \textit{output.out}\\
      \texttt{\#include <stdio.h> <unistd.h>\\
              int main()\{ \\
              \halftab for (int i = 0; i<3; i++) \{\\
              \tab sleep(2);\\
              \tab fprintf(stdout,"Written in buffer");\\
              \tab fflush(stdout);\\
              \halftab \};\\ \};}\par
      \textit{input.out}\\
      \texttt{\#include <stdio.h> <unistd.h>\\
              int main() \{\\
              \halftab char msg[50]; int n=3;\\
              \halftab while((n--)>0)\{\\
              \tab int c = read(0,msg,49); \\
              \tab if (c>0) \{ \\
              \tab \halftab msg[c]=0;\\
              \tab \halftab fprintf(stdout,"Read: '\%s' (\%d)\n",msg,c);\\
              \tab \};\\
              \halftab \};\\
              \};}
    \end{flushleft}
    Le pipe anonime, come quelle usate su shell, 'uniscono' due processi aventi un 
    antenato comune (oppure tra padre-figlio). Il collegamento \ace \textbf{unidirezionale} ed 
    avviene utilizzando un buffer di dimensione finita.\par 
    Per interagire con il buffer (la pipe) si usano \textbf{due} file descriptors: 
    uno per il lato in scrittura ed uno per il lato in lettura. Visto che i 
    processi figli ereditano i file descriptors, questo consente la comunicazione tra i processi.
    \begin{flushleft}
      \textbf{Creazione pipe}\par 
      \texttt{int pipe(int pipefd[2]); //fd[0] lettura, fd[1] scrittura}\par 
      \begin{flushleft}
        \texttt{\#include <stdio.h> <unistd.h>\\
                int main()\{\\
                \halftab int fd[2], cnt = 0;\\
                \halftab while(pipe(fd) == 0)\{ //Create unnamed pipe using 2 FD\\
                \tab cnt++;\\
                \tab printf("\%d,\%d,",fd[0],fd[1]);\\
                \halftab \}\\
                \halftab printf("\n Opened \%d pipes, then error\n",cnt);\\
                \halftab int op = open("/tmp/tmp.txt",O\_CREAT|O\_RDWR,S\_IRUSR|S\_IWUSR);\\
                \halftab printf("File opened with fd \%d\n",op);\\
                \}
        }
      \end{flushleft}
    \end{flushleft}
    \begin{flushleft}
      \textbf{Lettura pipe}\par 
      \texttt{int read(int fd[0],char * data, int num)}\\
      La lettura della pipe tramite il comando read restituisce valori differenti a seconda 
      della situazione:
      \begin{itemize}
        \item In caso di successo, read() restituisce il numero di bytes effettivamente letti
        \item Se il lato di scrittura \ace stato chiuso (da ogni processo), con il buffer vuoto 
              restituisce 0, altrimenti restituisce il numero di bytes letti.
        \item Se il buffer \ace vuoto ma il lato di scrittura \ace ancora aperto (in qualche processo) 
              il processo si sospende fino alla disponibilità dei dati o alla chiusura
        \item Se si provano a leggere pi\acu bytes (num) di quelli disponibili, vengono recuperati 
              solo quelli presenti
      \end{itemize}
      Esempio:
      \begin{flushleft}
        \texttt{\#include <stdio.h> <unistd.h>\\
        int main(void)\{ \\
        \halftab int fd[2]; char buf[50];\\
        \halftab int esito = pipe(fd); //Create unnamed pipe\\
        \halftab if(esito == 0)\{ \\
        \tab write(fd[1],"writing",8); // Writes to pipe\\
        \tab int r = read(fd[0],\&buf,50); //Read from pipe\\
        \tab printf("Last read \%d. Received: '\%s'\n",r,buf);\\
        \tab // close(fd[1]); // hangs when commented\\
        \tab r = read(fd[0],\&buf,50); //Read from pipe\\
        \tab printf("Last read \%d. Received: '\%s'\n",r,buf);\\
        \halftab \}\\ \}}
      \end{flushleft}
    \end{flushleft}
    \begin{flushleft}
      \textbf{Scrittura pipe}\par 
      \begin{flushleft}
        \texttt{int write(int fd[1],char * data, int num)}
      \end{flushleft}
      La scrittura della pipe tramite il comando write restituisce il numero di bytes scritti. 
      Tuttavia, se il lato in lettura \ace stato chiuso viene inviato un segnale \texttt{SIGPIPE} allo 
      scrittore (default handler quit) e la chiamata restituisce -1. \\
      In caso di scrittura, se vengono scritti \textbf{meno} bytes di quelli che ci possono stare 
      (\texttt{PIPE\_BUF}) la scrittura \ace "atomica" (tutto assieme), in caso contrario non c'\ace 
      garanzia di atomicit\aca e la scrittura sar\aca bloccata (in attesa che il buffer venga 
      svuotato) o fallir\aca se il flag \texttt{O\_NONBLOCK} viene usato.\\
      Per modificare le propriet\aca di una pipe, possiamo usare la system call \textbf{fnctl}, la 
      quale manipola i file descriptors.\\
      \begin{flushleft}
        \texttt{int fcntl(int fd, F\_SETFL, O\_NONBLOCK)}
      \end{flushleft}
      Esempio:
      \begin{flushleft}
        \texttt{\#include <unistd.h> <stdio.h> <signal.h> <errno.h> <stdlib.h> \\
                void handler(int signo)\{\\
                \halftab printf("SIGPIPE received\n"); perror("Error in handler"); \\
                \}\\
                int main(void)\{\\
                \halftab signal(SIGPIPE,handler);\\
                \halftab int fd[2]; char buf[50];\\
                \halftab int esito = pipe(fd); //Create unnamed pipe\\
                \halftab close(fd[0]); //Close read side\\
                \halftab printf("Attempting write\n");\\
                \halftab int numOfWritten = write(fd[1],"writing",8);\\
                \halftab printf("I've written \%d bytes\n",numOfWritten);\\
                \halftab perror("Error after write");\\
        \} }
      \end{flushleft}
    \end{flushleft}
    \begin{flushleft}
      \textbf{Comunicazione unidirezionale}\\
      Un tipico esempio di comunicazione unidirezionale tra un processo scrittore P1 ed un 
      processo lettore P2 \ace il seguente:
      \begin{itemize}
        \item P1 crea una pipe()
        \item P1 esegue un fork() e crea P2
        \item P1 chiude il lato lettura: close(fd[0])
        \item P2 chiude il lato scrittura: close(fd[1])
        \item P1 e P2 chiudono l'altro fd appena finiscono di comunicare.
      \end{itemize}
      \begin{flushleft}
        \texttt{\#include <stdio.h> <unistd.h> <sys/wait.h>\\
                int main()\{ \\
                \halftab int fd[2]; char buf[50];\\
                \halftab pipe(fd); //Create unnamed pipe\\
                \halftab int p2 = !fork();\\
                \halftab if(p2)\{\\
                \tab close(fd[1]);\\
                \tab int r = read(fd[0],\&buf,50); //Read from pipe\\
                \tab close(fd[0]); printf("Buf: '\%s'\n",buf);\\
                \halftab \}else\{\\
                \tab close(fd[0]);\\
                \tab write(fd[1],"writing",8); // Write to pipe\\
                \tab close(fd[1]);\\
                \halftab \}\\
                \halftab while(wait(NULL)>0);\\
        \}}
      \end{flushleft}
    \end{flushleft}
    \begin{flushleft}
      \textbf{Comunicazione bidirezionale}\par
      Un tipico esempio di comunicazione bidirezionale tra un processo scrittore P1 ed un 
      processo lettore P2 \ace il seguente:
      \begin{itemize}
        \item P1 crea due pipe(), pipe1 e pipe2
        \item P1 esegue un fork() e crea P2
        \item P1 chiude il lato lettura di pipe1 ed il lato scrittura di pipe2
        \item P2 chiude il lato scrittura di pipe1 ed il lato lettura di pipe2
        \item P1 e P2 chiudono gli altri fd appena finiscono di comunicare.
      \end{itemize}
      Esempio:
      \begin{flushleft}
        \texttt{\#include <stdio.h> <unistd.h> <sys/wait.h>\\
                \#define READ 0, WRITE 1 \\
                int main()\{ \\
                \halftab int pipe1[2] \\ 
                \halftab int pipe2[2];\\
                \halftab char buf[50];\\
                \halftab pipe(pipe1); \\
                \halftab pipe(pipe2); //Create two unnamed pipe\\
                \halftab int p2 = !fork();\\
                \halftab if(p2)\{ \\
                \tab close(pipe1[WRITE]); close(pipe2[READ]);\\
                \tab int r = read(pipe1[READ],\&buf,50); //Read from pipe\\
                \tab close(pipe1[READ]); printf("P2 received: '\%s'\n",buf);\\
                \tab write(pipe2[WRITE],"Msg from p2",12); // Writes to pipe\\
                \tab close(pipe2[WRITE]);\\
                \halftab \}else\{\\
                \tab close(pipe1[READ]); close(pipe2[1]);\\
                \tab write(pipe1[WRITE],"Msg from p1",12); // Writes to pipe\\
                \tab close(pipe1[WRITE]);\\
                \tab int r = read(pipe2[READ],\&buf,50); //Read from pipe\\
                \tab close(pipe2[READ]); printf("P1 received: '\%s'\n",buf);\\
                \halftab \} \\
                \halftab while(wait(NULL)>0);\\
                \}}
      \end{flushleft}
    \end{flushleft}
    \begin{flushleft}
      \textbf{Gestire la comunicazione}\par
      Per gestire comunicazioni complesse c'\ace bisogno di definire un "protocollo". Esempio:
      \begin{itemize}
        \item Messaggi di lunghezza fissa (magari inviata prima del messaggio)
        \item Marcatore di fine messaggio (per esempio con carattere NULL o newline)
      \end{itemize}
      Pi\acu in generale occorre definire la sequenza di messaggi attesi.
    \end{flushleft}
    \begin{flushleft}
      \textbf{Esempio: redirige lo stdout di cmd1 sullo stdin di cmd2}\\
      \texttt{\#include <stdio.h> <unistd.h> \\
      \#define READ 0, WRITE 1 \\
      int main (int argc, char *argv[]) \{\\
      \halftab int fd[2];\\
      \halftab pipe(fd); // Create an unnamed pipe \\
      \halftab if (fork() != 0) \{ // Parent, writer\\
      \tab close(fd[READ]); // Close unused end \\
      \tab dup2(fd[WRITE], 1); // Duplicate used end to stdout \\
      \tab close(fd[WRITE]); // Close original used end \\
      \tab execlp(argv[1], argv[1], NULL); // Execute writer program \\
      \tab perror("error"); // Should never execute \\
      \halftab \} else \{ // Child, reader \\
      \tab close(fd[WRITE]); // Close unused end \\
      \tab dup2(fd[READ], 0); // Duplicate used end to stdin \\
      \tab close(fd[READ]); // Close original used end\\
      \tab execlp(argv[2], argv[2], NULL); // Execute reader program \\
      \tab perror("error"); // Should never execute \\
      \halftab \}\\
      \}}
    \end{flushleft}
  \end{flushleft}
\end{flushleft}
\begin{flushleft}
  \subsection{FIFO/con nome}
  \textbf{Pipe con nome/FIFO}\\
  Le pipe con nome, o \textbf{FIFO}, sono delle pipe che corrispondono a dei file speciali nel 
  filesystem grazie ai quali i processi, senza vincoli di gerarchia, possono comunicare. \\
  Un processo pu\aco accedere ad una di queste pipe se ha i permessi sul file 
  corrispondente ed \ace vincolato, ovviamente, dall'esistenza del file stesso.\\
  Le FIFO sono interpretate come dei file, perci\aco si possono usare le funzioni di 
  scrittura/lettura dei file. Restano per\aco delle pipe, con i loro 
  vincoli e le loro capacità. \\ 
  \textbf{NB}: non sono dei file effettivi, quindi \textbf{lseek non funziona}, il 
  loro contenuto \ace sempre vuoto e non vi ci si pu\aco scrivere con un editor!\\
  Normalmente aprire una FIFO blocca il processo finch\ace anche l'altro lato non \ace stato
  aperto. Le differenze tra pipe anonime e FIFO sono solo nella loro creazione e gestione. 
  \begin{flushleft}
    \textbf{Creare una FIFO}\\
    \texttt{int mkfifo(const char *pathname, mode\_t mode);}\\
    Crea una FIFO (un file speciale), solo se non esiste gi\aca, con il nome \texttt{pathname}. Il 
    parametro mode pu\aco definire i privilegi del file nella solita maniera. 
    \begin{flushleft}
      Esempio: \\
      \texttt{\#include <sys/stat.h><sys/types.h><unistd.h><fcntl.h><stdio.h>\\
              int main(void)\{\\
              \halftab char * fifoName = "/tmp/fifo1";\\
              \halftab mkfifo(fifoName,S\_IRUSR|S\_IWUSR); //Create pipe if it !exist\\
              \halftab perror("Created?");\\
              \}}
    \end{flushleft}
  \end{flushleft}
  \begin{flushleft}
    \textbf{Comunicazione bloccata}\\
    \texttt{\#include <sys/stat.h><sys/types.h><unistd.h><fcntl.h><stdio.h>\\
            int main(void)\{\\
            \halftab char * fifoName = "/tmp/fifo1";\\
            \halftab mkfifo(fifoName,S\_IRUSR|S\_IWUSR); //Create pipe if it !exist\\
            \halftab perror("Created?");\\
            \halftab if (fork() == 0)\{ //Child\\
            \tab open(fifoName,O\_RDONLY); //Open READ side of pipe...stuck!\\
            \tab printf("Open read\n");\\
            \halftab \}else\{\\
            \tab sleep(3); \\
            \tab open(fifoName,O\_WRONLY); //Open WRITE side of pipe\\
            \tab printf("Open write\n");\\
            \halftab \}\\
            \}
            }
  \end{flushleft}
  \begin{flushleft}
    \textbf{Writer} \\
    \texttt{\#include<sys/stat.h><sys/types.h><unistd.h><fcntl.h><stdio.h><string.h>\\
            int main (int argc, char *argv[]) \{\\
            \halftab int fd; char * fifoName = "/tmp/fifo1"; \\
            \halftab char str1[80], * str2 = "I’m a writer";\\
            \halftab mkfifo(fifoName,S\_IRUSR|S\_IWUSR); //Create pipe if it !exist\\
            \halftab fd = open(fifoName, O\_WRONLY); // Open FIFO for write only\\
            \halftab write(fd, str2, strlen(str2)+1); // write and close\\
            \halftab close(fd);\\
            \halftab fd = open(fifoName, O\_RDONLY); // Open FIFO for Read only\\
            \halftab read(fd, str1, sizeof(str1)); // Read from FIFO\\
            \halftab printf("Reader is writing: \%s\n", str1);\\
            \halftab close(fd);
            \}}
  \end{flushleft}
  \begin{flushleft}
    \textbf{Reader}\\
    \texttt{\#include<sys/stat.h><sys/types.h><unistd.h><fcntl.h><stdio.h><string.h>\\
            int main (int argc, char *argv[]) \{\\
            \halftab int fd; char * fifoName = "/tmp/fifo1"; \\
            \halftab mkfifo(fifoName,S\_IRUSR|S\_IWUSR); //Create pipe if !exist\\
            \halftab char str1[80], * str2 = "I'm a reader";\\
            \halftab fd = open(fifoName , O\_RDONLY); // Open FIFO for read only\\
            \halftab read(fd, str1, 80); // read from FIFO and close it\\
            \halftab close(fd);\\
            \halftab printf("Writer is writing: \%s\n", str1);\\
            \halftab fd = open(fifoName , O\_WRONLY); // Open FIFO for write only\\
            \halftab write(fd, str2, strlen(str2)+1); // Write and close\\
            \halftab close(fd);\\
            \} }
  \end{flushleft}
  \begin{flushleft}
    \textbf{Comunicazione sul terminale}\\
    \ac{E} possibile usare le FIFO da terminale, leggendo e scrivendo dati tramite gli 
    operatori di redirezione. \\
    \texttt{echo "message for pipe" > /path/nameOfPipe\\
    cat /path/nameOfPipe}\\
    \textbf{NB}: non si possono scrivere dati usando editor di testo! Una volta consumati i dati, 
    questi non sono pi\acu presenti sulla fifo.
  \end{flushleft}
\end{flushleft}
\begin{flushleft}
  \textbf{Pipe anonime vs FIFO}\\
  \begin{tabular}{c|c|c}
    & \textbf{pipe} & \textbf{FIFO} \\
    \hline 
    Rappresentazione & Buffer & Buffer \\
    \hline
    Rifermento & anonimo & File \\
    \hline
    Accesso & 2 FileDescriptor & 1 FileDescriptor \\
    \hline
    Persistenza & Eliminazione alla  & Esistenza finch\ace \\ 
    & terminazione dei processi & c'\ace il file\\
    \hline
    Vincoli di accesso & antenato comune & Permessi su file\\
    \hline
    Creazione & pipe() & mkfifo() \\
    \hline
    MaxBytes per atomicit\aca & PIPE\_BUF = 4096 & PIPE\_BUF = 4096 \\
  \end{tabular}
\end{flushleft}
\begin{flushleft}
  \textbf{Conlusioni}\\
  PIPE e FIFO sono sistemi di comunicazione tra processi ("parenti", 
  tipicamente padre-figlio, nel primo caso e in generale nel secondo caso) che 
  consentono \textbf{scambi} di informazioni (\textbf{messaggi}) e \textbf{sincronizzazione} (grazie al fatto di 
  poter essere "bloccanti").
\end{flushleft}