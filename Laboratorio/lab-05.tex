\section{Lab-05}
\begin{flushleft}
  \textbf{Architettura}\\
  \begin{flushleft}
    \textbf{Kernel Unix}: il kernel \ace l'elemento di base di un sistema Unix-like, ovvero il nucleo 
    del sistema operativo. Il kernel \ace incaricato della gestione delle risorse essenziali (CPU, mem, periferiche) \\
    Ad ogni boot il sistema verifica lo stato delle periferiche, monta la prima partizione (roo file system) in read-only e 
    carica il kernel in memoria. \\ Il kernel lancia il primo programma (systemd) che a seconda 
    della configurazione voluta, ilizializza il sistema di conseguenza \\
    Il resto delle operazioni, tra cui l'interazione con l'utente, vengono gestite con programmi eseguiti dal kernel \par 
    \textbf{Kernel \& Mem Virtuale} \par 
    I programmi utilizzati dall'utente che vogliono accedere alle periferiche chedono al kernel di farlo per loro \\
    L'interazione tra programmi ed il resto del sistema viene mascherata da alcune caratteristiche intrinseche ai processori, come 
    la gestione del HW e della MemVirtuale \\
    Ogni programma vede se stesso come \textbf{unico processore della CPU} e npn gli \ace 
    dunque possibile disturbare l'azione degli altri programmi, $\Rightarrow$ stabilit\aca sistemi Unix \par 
    \textbf{Privilegi} \par 
    Nei sistemi Unix-like ci sono due livelli di privilegi:
    \begin{itemize}
      \item \textbf{User space}: ambiente in cui vengono eseguiti i programmi
      \item \textbf{Kernel space}: ambiente in cui viene eseguito il kernel
    \end{itemize}
  \end{flushleft}
  \begin{flushleft}
    \textbf{System calls} \par 
    Le interfacce con cui i programmi accedono al hw si chiamano \textbf{system calls}, cio\ace chiamate al sistema 
    che il kernel esegue nel \textbf{kernel space} retituendo i risultati al programma chiamante nello user space \\
    Le chiamate restituiscono -1 in caso di errore e settano la varaibile globale \texttt{errno}. Errori validi sono numeri 
    positivi e seguono lo standard POSIX, che definisce gli alias \par 
    \textbf{Librerie di sistema} \par 
    Utilizzando il comando di shell \texttt{ldd} su di un eseguibile si possono visualizzare
  \end{flushleft}
\end{flushleft}