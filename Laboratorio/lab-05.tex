\section{Lab-05}
\begin{flushleft}
  \textbf{Architettura}\\
  \begin{flushleft}
    \textbf{Kernel Unix}: il kernel \ace l'elemento di base di un sistema Unix-like, ovvero il nucleo 
    del sistema operativo. Il kernel \ace incaricato della gestione delle risorse essenziali (CPU, mem, periferiche) \\
    Ad ogni boot il sistema verifica lo stato delle periferiche, monta la prima partizione (roo file system) in read-only e 
    carica il kernel in memoria. \\ Il kernel lancia il primo programma (systemd) che a seconda 
    della configurazione voluta, ilizializza il sistema di conseguenza \\
    Il resto delle operazioni, tra cui l'interazione con l'utente, vengono gestite con programmi eseguiti dal kernel \par 
    \textbf{Kernel \& Mem Virtuale} \par 
    I programmi utilizzati dall'utente che vogliono accedere alle periferiche chedono al kernel di farlo per loro \\
    L'interazione tra programmi ed il resto del sistema viene mascherata da alcune caratteristiche intrinseche ai processori, come 
    la gestione del HW e della MemVirtuale \\
    Ogni programma vede se stesso come \textbf{unico processore della CPU} e npn gli \ace 
    dunque possibile disturbare l'azione degli altri programmi, $\Rightarrow$ stabilit\aca sistemi Unix \par 
    \textbf{Privilegi} \par 
    Nei sistemi Unix-like ci sono due livelli di privilegi:
    \begin{itemize}
      \item \textbf{User space}: ambiente in cui vengono eseguiti i programmi
      \item \textbf{Kernel space}: ambiente in cui viene eseguito il kernel
    \end{itemize}
  \end{flushleft}
  \subsection{System Calls}
  \begin{flushleft}
    \textbf{System calls} \par 
    Le interfacce con cui i programmi accedono al hw si chiamano \textbf{system calls}, cio\ace chiamate al sistema 
    che il kernel esegue nel \textbf{kernel space} retituendo i risultati al programma chiamante nello user space \\
    Le chiamate restituiscono -1 in caso di errore e settano la varaibile globale \texttt{errno}. Errori validi sono numeri 
    positivi e seguono lo standard POSIX, che definisce gli alias \par 
    \textbf{Librerie di sistema} \par 
    Utilizzando il comando di shell \texttt{ldd} su di un eseguibile si possono visualizzare le 
    librerie condivise caricate e, fra queste, vi sono tipicamente anche ld-linux.so, e 
    libc.so.
    \begin{itemize}
      \item \textbf{ld-linux.so}: quando un programma è caricato in memoria, il sistema operativo 
      passa il controllo a ld-linux.so anzichè al normale punto di ingresso 
      dell’applicazione. ld-linux trova e carica le librerie richieste, prepara il 
      programma e poi gli passa il controllo.
      \item \textbf{libc.so}: la libreria GNU C solitamente nota come glibc che contiene le funzioni 
      basilari più comuni.
    \end{itemize}
    \textbf{Esempi di chiamate di sistema} \par 
    \begin{itemize}
      \item \textbf{time(), ctime()} \par 
            \texttt{time\_t time( time\_t *second )\\
                    char * ctime( const time\_t *timeSeconds )}\par 
            \texttt{\#include <time.h> \\
                    \#include <stdio.h>\\
                    void main()\{ \\
                    \halftab time\_t theTime;\\
                    \halftab time\_t whatTime = time(\&theTime); //seconds since 1/1/1970\\
                    \halftab //Print date in Www Mmm dd hh:mm:ss yyyy \\
                    \halftab printf("Current time = \%s= \%d\textbackslash n", \\ \tab \tab ctime(\&whatTime),theTime); \\
                    \}} 
      \item \textbf{chdir(), getcwd()} \par 
            \texttt{int chdir( const char *path );\\
                    char * getcwd( char *buf, size\_t sizeBuf );}\par 
            \texttt{\#include <unistd.h> \\
                    \#include <stdio.h>\\
                    void main()\{ \\
                    \halftab char s[100]; \\
                    \halftab getcwd(s,100); // copy path in buffer \\
                    \halftab printf("\%s\n", s); //Print current working dir \\
                    \halftab chdir(".."); //Change working dir \\
                    \halftab printf("\%s\n", getcwd(NULL,100)); // Allocates buffer\\
                    \}}
      \item \textbf{dup(), dup2()}\par 
            \texttt{int dup(int oldfd);\\
                    int dup2(int oldfd, int newfd);}  \par
            \texttt{\#include <unistd.h> <stdio.h> <fcntl.h>\\
                    int main(void)\{\\
                    \halftab char buf[51];\\
                    \halftab int fd = open("file.txt",O\_RDWR); //file exists\\
                    \halftab int r = read(fd,buf,50); //Read 50 bytes from ‘fd’ in ‘buf’\\
                    \halftab buf[r] = 0; printf("Content: \%s\n",buf);\\
                    \halftab int cpy = dup(fd); // Create copy of file descriptor\\
                    \halftab \textit{// Copy cpy to descriptor 22 (close 22 if opened)} \\
                    \halftab dup2(cpy,22); \\ 
                    \halftab \textit{// Move I/O on all 3 file descriptors!} \\
                    \halftab lseek(cpy,0,SEEK\_SET); \\
                    \halftab \textit{// Write starting from 0-pos} \\
                    \halftab write(22,"This is a fine\n",16); \\
                    \halftab close(cpy); //Close ONE file descriptor
                    \} }
      \item \textbf{chmod(), chown()} \par 
            \texttt{int chown(const char *pathname, uid\_t owner, gid\_t group)\\
                    int fchown(int fd, uid\_t owner, gid\_t group)\\
                    int chmod(const char *pathname, mode\_t mode)\\
                    int fchmod(int fd, mode\_t mode)}\par  
            \texttt{\#include <fcntl.h>\\
                    \#include <unistd.h>\\
                    \#include <sys/stat.h>\\
                    void main()\{ \\
                    \halftab int fd = open("file",O\_RDONLY);\\
                    \halftab // Change owner to user:group 1000:1000\\
                    \halftab fchown(fd, 1000, 1000); \\
                    \halftab // Permission to r/r/r\\
                    \halftab chmod("file",S\_IRUSR|S\_IRGRP|S\_IROTH); \\
                    \}}
    \end{itemize}
    \subsection{Exec()}
    \textbf{Exec() family} \par 
    La famiglia di funzioni exec ha come scopo finale l'esecuzione di un programma, 
    sostituendo l'immagine del processo corrente con una nuova immagine. \textbf{Il PID del
    processo e la sua file table non cambiano}.\\
    La chiamata di sistema di base \ace execve(), ma vedremo tutti i suoi alias che 
    differiscono solo per gli argomenti 
    accettati, mantenendo lo stesso identico comportamento. \\
    Ogni alias \ace composto dalla parola chiave exec seguita dalle seguenti lettere:
    \begin{itemize}
      \item l: accetta una lista di argomenti
      \item v: accetta un vettore, quindi un solo argomento di diversi argomenti
      \item p: usa la variabile d'ambiente PATH per cercare il binario
      \item e: usa un vettore di variabili d'ambiente (es. "name=value") 
    \end{itemize}
    NB: Ogni vettori di argomenti deve terminare con un elemento NULL\par 
    \begin{flushleft}
      \textbf{Chiamte funzioni}: 
      \begin{itemize}
        \item \texttt{int execv (const char *path, char *const argv[])}
        \item \texttt{int execvp (const char *file, char *const argv[])}
        \item \texttt{int execve (const char *path, char *const argv[],
                      char *const envp[]);}
        \item \texttt{int execvpe (const char *file, char *const argv[],
                      char *const envp[])}
        \item \texttt{int execl (const char *path, const char * arg0,…,argn,NULL)}
        \item \texttt{int execlp (const char *file, const char * arg0,…,argn,NULL)}
        \item \texttt{int execle (const char *path, const char * arg0,…,argn,NULL,
                      char *const envp[])}
        \item \texttt{int execlpe(const char *file, const char * arg0,…,argn,NULL, 
                      char *const envp[])}
      \end{itemize}
      \textbf{Esempi}:
      \begin{flushleft}
        \begin{itemize}
          \item \textbf{execv()}\par 
                \textit{execv1.out} \\
                \texttt{\#include <unistd.h>\\
                        \#include <stdio.h>\\
                        void main()\{ \\
                        \halftab char * argv[] = \{"par1","par2",NULL\};\\
                        \halftab execv("./execv2.out",argv); //Replace current process \\
                        \halftab printf("This is execv1\n"); \\
                        \}}\par 
                \textit{execv2.out} \\
                \texttt{\#include <stdio.h> \\
                        void main(int argc, char ** argv) \{ \\
                        \halftab printf("This is execv2 with \%s and \%s\n",argv[0],argv[1]);\\  
                        \} }
          \item \textbf{execle()} \par 
                \textit{execle1.out} \\
                \texttt{\#include <unistd.h>\\
                        \#include <stdio.h> \\
                        void main()\{ \\
                        \halftab char * env[] = {"CIAO=hello world",NULL};\\
                        \halftab //Replace proc.\\
                        \halftab execle("./execle2.out","par1","par2",NULL,env); \\ 
                        \halftab printf("This is execle1\n"); \\
                        \}} \par 
                \textit{execle2.out} \\
                \texttt{\#include <stdio.h> \\
                        \#include <stdlib.h>\\
                        void main(int argc, char ** argv)\{ \\
                        \halftab printf("This is execle2 with par:\%s and \%s. \\ \tab CIAO =
                                  \%s\n",argv[0],argv[1],getenv("CIAO")); \\
                        \}} 
          \item \textbf{dup2/exec} \par 
                \textit{execvDuo.o} \\
                \texttt{\#include <stdio.h> \\
                        \#include <fcntl.h> \\
                        \#include <unistd.h>\\
                        void main() \{ \\
                        \halftab int outfile = open("/tmp/out.txt",
                        O\_RDWR | O\_CREAT, \\ \tab \tab \tab \tab S\_IRUSR | S\_IWUSR); \\
                        \halftab dup2(outfile, 1); // copy outfile to FD 1 \\
                        \halftab char *argv[]=\{"./time.out",NULL\}; // time.out esempi prec \\
                        \halftab execvp(argv[0],argv); // Replace current process\\
                        \}} 
        \end{itemize}
      \end{flushleft}

    \end{flushleft}
    \begin{flushleft}
      \textbf{system()} \par 
      \texttt{int system(const char * str);} \\
      \texttt{\#include <stdlib.h> <stdio.h> \\
              \#include <sys/wait.h> /* For WEXITSTATUS */ \\
              void main()\{ \\
              \halftab int outcome = system("echo ciao"); // execute command in shell \\
              \halftab printf("Outcome = \%d\n",outcome); \\
              \halftab outcome = system("if [[ \$PWD < \"ciao\" ]]; then echo min; fi"); \\
              \halftab printf("Outcome = \%d\n",outcome); \\
              \halftab outcome = system("notExistingCommand"); \\
              \halftab printf("Outcome = \%d\n",WEXITSTATUS(outcome)); \\
              \} }
    \end{flushleft}
  \end{flushleft}
  \subsection{Fork()}
  \begin{flushleft}
    \textbf{Forking} \par 
    Il forking \ace la "generazione" di nuovi processi (uno alla volta) partendo da uno 
    esistente. La syscall principale per il forking \ace \texttt{fork()}. \\
    Quando un processo attivo invoca questa syscall, il kernel lo clona modificando 
    per\aco alcune informazioni, in particolare quelle che riguardano la sua collocazione 
    nella gerarchia complessiva dei processi. \\
    Il processo che effettua la chiamata \ace definito "padre/genitore", quello generato \ace 
    definito "figlio". \par 
    \textbf{Identificativi dei processi}\\
    Ad ogni processo \ace associato un identificativo univoco per istante temporale, sono 
    organizzati gerarchicamente (genitore-figlio) e suddivisi in insiemi principali 
    (sessioni) e secondari (gruppi). Anche gli utenti hanno un loro identificativo e ad 
    ogni processo ne sono abbinati due: quello reale e quello effettivo (di esecuzione)
    \begin{itemize}
      \item PID - Process ID
      \item PPID - Parent Process ID
      \item SID - Session ID
      \item PGID - Process Group ID
      \item UID/RUID - (Real) User ID
      \item EUID - Effective User ID
    \end{itemize}
    \textbf{fork: elementi clonati e elementi nuovi} \par 
    Sono clonati gli elementi principali come il PC (Program Counter), i registri, la 
    tabella dei file (file descriptors) e i dati di processo (variabili). Le meta-informazioni 
    come il "pid" e il "ppid" sono aggiornate (non come in execve()). \\
    L'esecuzione procede per entrambi (quando saranno schedulati) da PC+1 
    (tipicamente l'istruzione seguente il fork o la valutazione dell'espressione in cui essa 
    è utilizzata):
    \begin{itemize}
      \item \texttt{fork();\\ printf("\n");} \\
            Prossimo step: printf
      \item \texttt{f = fork(); \\ printf("\n")} \\
            Prossimo step: assegnamento ad f
    \end{itemize}
    \textbf{getpid(), getppid()} 
    \begin{itemize}
      \item \texttt{pid\_t getpid()} : restituisce il PID del processo attivo
      \item \texttt{pid\_t getppid()} : restituisce il PID del processo genitore
    \end{itemize}
    \textbf{Esempio}: \\ 
    \texttt{\#include <stdio.h> <unistd.h> <stdlib.h> \\
            void main()\{ \\
            \halftab printf("Subshell \$\$ = "); \\
            \halftab fflush(stdout); // Forza l'output di printf\\
            \halftab system("echo \$\$"); // subshell\\
            \halftab printf("PID: \%dPPID: \%d\n",getpid(),getppid());\\
            \}}
    N.B.: includendo le librerie \texttt{<sys/types.h>} e \texttt{sys/wait.h} il tipo \texttt{pid\_t} 
    \ace un intero che rappresenta un id del processo \par  
    \textbf{fork: valore di ritorno} \par
    La funzione restituisce un valore che solitamente \ace catturato in una variabile (o usato 
    comunque in un'espressione). \\
    Come per tutte le syscall in generale, il valore è -1 in caso di errore (in questo caso 
    non ci sar\aca nessun nuovo processo, ma solo quello che ha invocato la chiamata). \\
    Se ha successo entrambi i processi ricevono un valore di ritorno, ma questo \ace diverso
    nei due casi:
    \begin{itemize}
      \item Il processo genitore riceve come valore il nuovo PID del processo figlio
      \item Il processo figlio riceve come valore 0
    \end{itemize}
    \textbf{fork: relazione tra i processi} \par
    I processi genitore-figlio:
    \begin{itemize}
      \item Conoscono reciprocamente il loro PID (ciascuno conosce il proprio tramite 
            getpid(), il figlio conosce quello del genitore con getppid(), il genitore 
            conosce quello del figlio come valore di ritorno di fork())
      \item Si possono usare altre syscall per semplici interazioni come wait e waitpid
      \item Eventuali variabili definite prima del fork sono valorizzate allo stesso modo in 
            entrambi: se riferiscono risorse (ad esempio un “file descriptor” per un file su 
            disco) fanno riferimento esattamente alla stessa risorsa
    \end{itemize}
    \textbf{fork: wait()} \par 
    \texttt{pid\_t wait (int *status)} \\
    \begin{flushleft}
      Attende il cambio di stato di un processo figlio (uno qualsiasi) restituendone il PID e salvando 
      in status lo stato del figlio (se il puntatore non \ace NULL). Il cambio di stato avviene se il figlio 
      viene terminato o la sua esecuzione interrotta/ripresa a seguito di un segnale \\
      Se non esiste alcun figlio restituisce -1. \\
      Nel nostro caso ci interessa principalmente la terminazione del figlio. Questa system call ci 
      permette di bloccare il processo (anche sincronizzarlo) fino a quando il figlio non ha finito le sue 
      operazioni. \par
      \texttt{while(wait(NULL)>0);} questo comando aspetta tutti i figli \par 
      \textbf{Interpretazione stato}\\
      Lo stato di ritorno \ace un numero che comprende più valori "composti" interpretabili con apposite 
      macro, molte utilizzabili a mo' di funzione (altre come valore) passando lo "stato" ricevuto come 
      risposta come ad esempio:
      \begin{itemize}
        \item \texttt{WEXITSTATUS(sts)}: restituisce lo stato vero e proprio (ad esempio il valore usato nella “exit”).
        \item \texttt{WIFCONTINUED(sts)}: true se il figlio ha ricevuto un segnale SIGCONT.
        \item \texttt{WIFEXITED(sts)}: true se il figlio è terminato normalmente.
        \item \texttt{WIFSIGNALED(sts)}: true se il figlio è terminato a causa di un segnale non gestito.
        \item \texttt{WIFSTOPPED(sts)}: true se il figlio è attualmente in stato di “stop”.
        \item \texttt{WSTOPSIG(sts)}: numero del segnale che ha causato lo “stop” del figlio.
        \item \texttt{WTERMSIG(sts)}: numero del segnale che ha causato la terminazione del figlio.
      \end{itemize}
      \textbf{Esempio}
      \texttt{\#include <stdio.h> <unistd.h> <sys/wait.h> \\
              int main(void) \{ \\
              \halftab int isChild = !fork(); \\
              \halftab if(isChild) \{ \\
              \tab sleep(3); return 5; \\
              \halftab \} \\
              \halftab int childStatus; \halftab wait(\&childStatus); \\
              \halftab printf("Children terminated? \%d\n Return code: \%d\n",
              \tab WIFEXITED(childStatus),WEXITSTATUS(childStatus)); \\
              \halftab return 0;
              \}}\\
      \begin{flushleft}
      \textbf{fork: waitpid()} \par 
        \texttt{pid\_t waitpid(pid\_t pid, int *status, int options)}
      \end{flushleft}
      Consente un'attesa selettiva basata su dei parametri. \textbf{pid} pu\ace essere:
      \begin{itemize}
        \item -n (aspetta un figlio qualsiasi nel gruppo $|-n|$)
        \item -1 (aspetta un figlio qualsiasi)
        \item 0 (aspetta un figlio qualsiasi appartenente allo stesso gruppo)
        \item n (aspetta il figlio con PID=n)
      \end{itemize}
      \textbf{options} sono i seguenti parametri ORed:
      \begin{itemize}
        \item WNOHANG: ritorna immediatamente se nessun figlio \ace terminato allora non si resta in attesa
        \item WUNTRACED: ritorna anche se un figlio si \ace interrotto senza terminare.
        \item WCONTINUED: ritorna anche se un figlio ha ripreso l'esecuzione.
      \end{itemize}
      \texttt{wait(st)} \ace l'equivalente di \texttt{waitpid(-1, st, 0)} 
    \end{flushleft}
    \begin{flushleft}
      \textbf{fork multiplo}: \ace possibile effettuare pi\acu fork all'interno dello stesso programma
    \end{flushleft}
    \begin{flushleft}
      \textbf{Esempio fork \& wait} \\
      \texttt{\#include <stdio.h> <stdlib.h> <unistd.h> <time.h> <sys/wait.h>\\
              int main() \{\\
              \halftab int fid=fork(), wid, st, r; // Generate child\\
              \halftab srand(time(NULL)); // Initialise random\\
              \halftab r=rand()\%256; // Get random between 0 and 255\\
              \halftab if (fid==0) \{ //If it is child\\
              \tab printf("Child... (\%d)", r); fflush(stdout);\\
              \tab sleep(3); // Pause execution for 3 seconds\\
              \tab printf(" done!\n");\\
              \tab exit(r); // Terminate with random signal\\
              \halftab \} else \{ // If it is parent\\
              \tab printf("Parent...\n");\\
              \tab wid=wait(\&st); // wait for ONE child to terminate\\
              \tab printf("...child's id: \%d==\%d (st=\%d)\n", fid, wid, \\ \tab \tab WEXITSTATUS(st));\\
              \halftab \} \\  \}}
    \end{flushleft}
    \begin{flushleft}
      \textbf{Processi zombie e orfani} \\
      Normalmente quando un processo termina il suo stato di uscita viene "catturato" dal 
      genitore: alla terminazione il sistema tiene traccia di questo insieme di informazioni 
      (lo stato) fino a che il genitore le utilizza consumandole (con wait o waitpid). \\
      Se il genitore non cattura lo stato d'uscita, i suoi processi figli vengono definiti \textbf{zombie} 
      (in realt\ace non ci sono pi\acu, ma esiste un riferimento in sospeso nel sistema). \\
      Se un genitore termina prima del figlio, quest'ultimo viene definito \textbf{orfano} e viene 
      adottato dal processo principale (tipicamente “systemd” con pid pari a 1). \\
      Un processo zombie che diventa anche orfano \ace poi gestito dal processo che lo adotta 
      (che effettua periodicamente dei wait/waitpid appositamente). \\
      Per ispezionare la lista di processi attivi usare il comando \texttt{ps} con le seguenti 
      opzioni:
      \begin{itemize}
        \item a: mostra lo stato (T: stopped, Z: zombie, R: running, etc…)
        \item -H: mostra la gerarchia processi
        \item -e: mostra l'intera lista dei processi, non solo della sessione corrente
        \item -f: mostra il PID del padre
      \end{itemize}
    \end{flushleft}
  \end{flushleft}
\end{flushleft}